\documentclass[14pt]{extarticle}

\usepackage[T2A]{fontenc}
\usepackage[utf8]{inputenc}
\usepackage[russian]{babel}

\usepackage{graphicx}

\usepackage{amsmath, amsthm, amsfonts, amssymb}
\usepackage[unicode, hidelinks]{hyperref}

\usepackage[
    a4paper, mag=1000,
    left=2.5cm, right=1.5cm, top=2cm, bottom=2cm, bindingoffset=0cm,
    headheight=0cm, footskip=1cm, headsep=0cm
]{geometry}

\usepackage{tempora}

\graphicspath{{images/}}

\begin{document}

\tableofcontents

\newpage

\section{Что это такое?}

В этом файле содержится информация по дискретной математике,
которая, по моему мнению, поможет в понимании материала по дискретной
математике и прольёт свет на некоторые используемые в ответах на билеты
термины. Кроме того, эта информация поможет лучше подготовиться к экзамену
и почувствовать себя уверенней.

\section{Теория по графам}

\subsection{Определение порядка графа:}

Число $|V|$ вершин графа $G$ называется его порядком.

\subsection{Определение смежных вершин:}

Две вершины называются смежными, если они соединены ребром.

\subsection{Определение смежных рёбер:}

Два ребра называются смежными, если они имеют общую вершину.

\subsection{Определение матрицы смежности графа:}

Матрица смежности~--- квадратная матрица $A = (a_{ij}), \;\;\; i, j = \overline{1, p}$,
где 

\[
    a_{ij} =
    \begin{cases}
        1, (i, j) \in \rho \\
        0, (i, j) \notin \rho\\
    \end{cases}  
\]

Запись $(i, j) \in \rho$ означает, что между вершинами $i$ и $j$ существует 
ребро.

\begin{center}
    \includegraphics[width=60mm]{"2.4.1.png"}
    \includegraphics[width=50mm]{"2.4.2.png"}
\end{center}

\subsection{Определение степени вершины в неориентированном графе:}

Степенью вершины $deg(v)$ в неориентированном графе называется число рёбер,
непосредственно соединённых с ней.

\begin{center}
    \includegraphics[width=130mm]{"2.5.1.png"}
\end{center}

\subsection{Определение чётности и нечётности вершины графа:}

Вершина графа называется четной, если ее степень четна,
и нечетной в противном случае.

\subsection{Определение вершинного вектора графа:}

Вершинным вектором графа называется вектор $(d_1, \dots, d_n)$,
где $d_1, \dots, d_n$~--- степени вершин графа.

\begin{center}
    \includegraphics[width=130mm]{"2.7.1.png"}
\end{center}

\end{document}