\documentclass[14pt]{extarticle}

\usepackage[T2A]{fontenc}
\usepackage[cp1251]{inputenc}
\usepackage[russian]{babel}

\usepackage{graphicx}

\usepackage{amsmath, amsthm, amsfonts, amssymb}
\usepackage[unicode, hidelinks]{hyperref}

\usepackage[
    a4paper, mag=1000,
    left=2.5cm, right=1.5cm, top=2cm, bottom=2cm, bindingoffset=0cm,
    headheight=0cm, footskip=1cm, headsep=0cm
]{geometry}

\usepackage{tempora}

\usepackage{chngcntr} % equation counter manipulations

% reset equation counter at each section
\counterwithin*{equation}{section}
\counterwithin*{equation}{subsection}

\graphicspath{{images/}}

\usepackage{cases}

\begin{document}
    
\tableofcontents

\newpage

\section{Комбинаторика, правило суммы и произведения.
Размещения с повторениями и без повторений.}

\textbf{Правило суммы:}
    \smallskip

    Если объект $A$ можно выбрать m способами, а объект $B$, после выбора $A$, 
    можно выбрать n способами, то пару $(A, B)$ иожно выбрать $n \times m$
    способами.
    \bigskip

\textbf{Правило произведения:}
    \smallskip

    Если $A$ можно выбрать n способами, а $B$ m способами, то объект 
    $A$ или $B$ можно выбрать $n + m$ способами. (Выбор $B$ никак не
    согласуется с выбором $A$.)
    \bigskip

\textbf{Размещения с повторениями:}
    \smallskip
    
    Размещениями с повторениями из n типов по k элементов (k и n в произвольном
    соотношении) называются все такие последовательности k элементов,
    принадлижащих n типам, которые отличаются друг от друга составом
    или последовательностью элементов. 
    
    \[
        \overline{A^{k}_n} = n^k 
    \]
    \bigskip

\textbf{Размещения без повторений:}
    \smallskip
    
    Размещениями без повторений из n различных типов по k элементам называются
    все такие последовательности из k различных элементов, такие, что они
    различаются по составу или по порядку. Причём k < n.

    \[
        A^k_n = \frac{n!}{(n - k)!}  
    \]

\end{document}