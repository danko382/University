\documentclass[14pt]{extarticle}

\usepackage[T2A]{fontenc}
\usepackage[cp1251]{inputenc}
\usepackage[russian]{babel}

\usepackage{graphicx}

\usepackage{amsmath, amsthm, amsfonts, amssymb}
\usepackage[unicode, hidelinks]{hyperref}

\usepackage[
    a4paper, mag=1000,
    left=2.5cm, right=1.5cm, top=2cm, bottom=2cm, bindingoffset=0cm,
    headheight=0cm, footskip=1cm, headsep=0cm
]{geometry}

\usepackage{tempora}

\usepackage{chngcntr} % equation counter manipulations

% reset equation counter at each section
\counterwithin*{equation}{section}
\counterwithin*{equation}{subsection}

\graphicspath{{images/}}

\usepackage{cases}

\begin{document}
    
\tableofcontents

\newpage

\section{Комбинаторика, правило суммы и произведения.
Размещения с повторениями и без повторений.}

\textbf{Правило суммы:}
    \smallskip

    Если $A$ можно выбрать $n$ способами, а $B$~--- $m$ способами, то объект 
    $A$ или $B$ можно выбрать $n + m$ способами. (Выбор $B$ никак не
    согласуется с выбором $A$.)
    \bigskip

\textbf{Правило произведения:}
    \smallskip

    Если объект $A$ можно выбрать $m$ способами, а объект $B$, после выбора $A$, 
    можно выбрать $n$ способами, то пару $(A, B)$ можно выбрать $n \times m$
    способами.
    \bigskip

\textbf{Размещения с повторениями:}
    \smallskip
    
    Размещениями с повторениями из $n$ типов по $k$ элементов ($k$ и $n$ в произвольном
    соотношении) называются все такие последовательности $k$ элементов,
    принадлижащих $n$ типам, которые отличаются друг от друга составом
    или последовательностью элементов. 
    
    \[
        \overline{A^{k}_n} = n^k 
    \]
    \bigskip

\textbf{Размещения без повторений:}
    \smallskip
    
    Размещениями без повторений из $n$ различных типов по $k$ элементам называются
    все такие последовательности из $k$ различных элементов, такие, что они
    различаются по составу или по порядку. Причём $k < n$.

    \[
        A^k_n = \frac{n!}{(n - k)!}  
    \]

\section{Перестановки с повторениями и без повторений. Сочетания с повторениями 
и без повторений, свойства биномиальных коэффициентов.}

\textbf{Перестановки с повторениями:}
    \smallskip

    Перестановками с повторениями из $n_1, \dots, n_k$ элементов $k$"=го типа 
    называются всевозможные последовательности длины $n$, отличающиеся друг от
    друга последовательностью элементов.

    \[
        \overline{P}(n_1, \dots, n_k) = \frac{n!}{n_1! \cdot \dots \cdot n_k!}  
    \]
    \bigskip

\textbf{Перестановски без повторений:}
    \smallskip
    
    Перестановками без повторений из $n$ элементов называются всевозможные
    последовательности из $n$ элементов.

    \[
        P_n = n!  
    \]
    \bigskip

\textbf{Сочетания с повторениями:}
    \smallskip
    
    Сочетаниями с повторениями из $n$ по $k$ ($k$ и $n$ в произвольном соотношении)
    называются все такие комбинации из $k$ элементов $\in n$ типам,
    которые отличаются только составом элементов.

    \[
        \overline{C^k}_n = C^k_{n + k - 1} = \overline{P}(n - 1, k) 
    \]
    \bigskip

\textbf{Сочетания без повторений:}
    \smallskip
    
    Сочетаниями без повторений из $n$ по $k$ ($k \leq n$) называются
    все такие комбинации из $k$ различных элементов, выбранных из
    $n$ исходных элементов, которые отличаются друг от друга составом.

    \[
        C^k_n = \frac{n!}{(n - k)! k!}  
    \]
    \bigskip

\textbf{Свойства биномиальных коэффициентов:}
    \smallskip
    
    \begin{enumerate}
        \item{
            \[
                C^k_n = \overline{P}(k, n - k)  
            \]
        }
        \item{
            \[
                C^k_n = C^{n - k}_n  
            \]
        }
        \item{
            \[
                C^k_n = C^{k - 1}_{n - 1} + C^{k}_{n - 1}  
            \]
        }
        \item{
            \[
                C^0_n + C^1_n + \dots + C^n_n = 2^n  
            \]
        }
        \item{
            \[
                C^0_n - C^1_n + C^2_n - C^3_n + \dots + C^n_n = 0  
            \]
        }
    \end{enumerate}

\section{Сколькими способами можно разложить $n_1$ предметов одного сорта,
$\dots, n_k$ предметов $k$"=го сорта в два ящика? Следствия.}

\textbf{Схема: $n_1$ предметов $1$"=го типа $\dots$ $n_k$ предметов $k$"=го типа
раскладываются в два различных ящика:}
    \smallskip

    \begin{center}
        $(n_1 + 1) \cdot (n_2 + 1) \cdot \dots \cdot (n_k + 1)$ способов.
    \end{center}
    \bigskip
 
\textbf{Следствие 1:}
    \smallskip
    
    Если все предметы различны, то:

    \begin{center}
        $n_1 = 1 = n_2 = \dots = n_k = 1 \Rightarrow 2^k$ способов.
    \end{center}
    \bigskip

\textbf{Следствие 2:}
    \smallskip
    
    Не менее $r_i$ предметов $i$"=го типа в каждый ящик:

    \begin{center}
        $(n_1 - 2r_1 + 1) \cdot (n_2 - 2r_2 + 1) \cdot \dots 
        \cdot (n_k - 2r_k + 1)$ способов.
    \end{center}

\section{Даны $n$ различных предметов и $k$ ящиков. Требуется положить в первый
ящик $n_1$ предметов, в $k$"=ый~--- $n_k$ предметов, где $n_1 + \dots + n_k = n$.
Сколькими способами можно сделать такое распределение, если не интересует порядок
распределения предметов в ящике?}    

\textbf{Схема: $n$ различных предметов раскладываются в $k$ различных
ящиков (порядок внутри ящиков не важен):}
    \smallskip

    \begin{center}
        $\frac{n!}{n_1! \cdot \dots \cdot n_k!}$ способов
    \end{center}

\section{Даны $n$ различных предметов и $k$ одинаковых ящиков. Требуется положить в 
каждый ящик $n = \frac{n}{k}$ предметов. Сколькими способами можно сделать такое 
распределение, если не интересует порядок предметов в ящике и все ящики 
одинаковы?}    

\textbf{Схема: $n$ различных предметов в $k$ одинаковых ящиков
(порядок внутри ящиков не важен) $\frac{n}{k}$ предметов в каждый ящик:}
    \smallskip

    \begin{center}
        $\frac{n!}{k!((\frac{n}{k})!)^{k}}$ способов.
    \end{center}

\section{Сколькими способами можно распределить $n$ одинаковых предметов в $k$ ящиков?}    

\textbf{Схема: $n$ одинаковых предметов в $k$ разных ящиков:}
    \smallskip

    \begin{center}
        $\overline{P}(n, k-1) = \frac{(n + k - 1)!}{n!(k - 1)!}$ способов.
    \end{center}

\section{Сколько существует способов разложить n различных  предметов в k ящиков, если 
нет никаких ограничений?}    

\textbf{Схема: $n$ различных предметов в $k$ разных ящиков:}
    \smallskip

\begin{center}
    $k^n$ способов.
\end{center}

\section{Сколькими способами можно положить  n различных предметов в k ящиков, если 
не должно быть пустых ящиков?}

\textbf{Схема: $n$ различных предметов в $k$ разных ящиков, причём не должно
быть пустых ящиков:}
    \smallskip

\begin{center}
    $A_i$~--- количество способов, когда $i$"=ый ящик пустой.
    \bigskip

    $|A \backslash \cup^{k}_{i = 1} A_i| = |A| - \sum^{k}_{i = 1} |A_i| + \sum |A_i \cap A_j| +
    \dots + (-1)^{k - 1} \sum_{i_1 \dots i_{k - 1}} |A_{i_1} \cap \dots \cap A_{i_{k - 1}}| +
    (-1)^{k} |A_1 \cap \dots \cap A_k| = k^n - C^1_k (k - 1)^n + C^2_k(k - 2)^n +
    \dots + (-1)^{k - 1} C^{k - 1}_k \cdot 1^n$ способов.
\end{center}

\section{Имеется $n_1$ предметов одного сорта$, \dots, n_s$~---$s$"=го сорта.
Сколькими способами их можно разложить по $k$ ящикам, если не должно быть
пустых ящиков?}

\textbf{Схема $n_1$ предметов первого типа $\dots n_m$ предметов $m$"=го типа
по $k$ различным ящикам, причём нет пустых ящиков:}
    \smallskip

    $A_i$~--- $i$"=ый ящик пустой $i = \overline{1, k}$.
    \bigskip

    $|A| = C^{k - 1}_{n_1 + k - 1} \cdot C^{k - 1}_{n_2 + k - 1} \cdot \dots
    \cdot C^{k - 1}_{n_m + k - 1}$
    \bigskip

    $|A_i| = C^{k - 2}_{n_1 + k - 2} \cdot C^{k - 2}_{n_2 + k - 2} \cdot \dots
    \cdot C^{k - 2}_{n_m + k - 2}$
    \bigskip

    $|A \backslash \cup^{k}_{i = 1} A_i| = |A| - \sum\limits^n_{i = 1} |A_i| +
    \sum |A_{i_1} \cap A_{i_2}| + \dots + (-1)^{k - 1} \sum\limits_{1 \leq i_1 \leq
    \dots \leq i_{k - 1} \leq k} |A_{i_1} \cap \dots \cap A_{i_{k - 1}}| +
    (-1)^{k} |A_1 \cap \dots \cap A_k| =$
    \bigskip

    $ = C^{k - 1}_{n_1 + k - 1} \cdot \dots
    \cdot C^{k - 1}_{n_m + k - 1} - C^1_{k} C^{k - 2}_{n_1 + k - 2} \cdot \dots
    \cdot C^{k - 2}_{n_m + k - 2} + C^2_k C^{k - 3}_{n_1 + k - 3} \cdot \dots
    \cdot C^{k - 3}_{n_m + k - 3} + \dots + (-1)^{k - 1} C^{k - 1}_{k} 1^n$.

\section{Сколько существует способов разложить $n$ различных  предметов в $k$  различных 
ящиков с учетом расположения предметов в ящиках, если все $n$ предметов должны 
быть использованы? Следствие.}

\textbf{Схема: $n$ различных предметов в $k$ различных ящиков (порядок внутри
ящиков важен, ящики могут быть пустыми):}
    \smallskip

    \begin{center}
        $\overline{P}(n, k-1) = \frac{(n + k - 1)!}{(k - 1)!} =
        A^n_{n + k - 1}$,
    \end{center}

    \bigskip

\textbf{Следствие (та же схема, но не должно быть пустых ящиков):}
    \smallskip    
    
    \[
        A^k_n A^{n-k}_{n-k+k-1} = A^k_n A^{n-k}_{n-1} = \frac{n!}{(n-k)!} =
        = \frac{(n-1)!}{(k-1)!} = n! C^{k-1}_{n-1}    
    \]

\section{Сколько существует способов разложить $n$ различных  предметов в $k$  различных 
ящиков с учетом расположения предметов в ящиках, если не все $n$ предметов могут 
быть использованы и некоторые ящики могут оказаться пустыми? Следствие.}    

\textbf{Схема: $n$ различных предметов в $k$ различных ящиков (порядок
внутри ящиков важен, можно использовать не все предметы):}
    \smallskip

    $S$~--- в распределении учавствует $s$ предметов. $S = \overline{0, n}$.
    \bigskip

    \[
        \sum\limits^n_{S=0} C^S_n A^S_{S + k - 1}  
    \]
    \bigskip

\textbf{Следствие (та же схема, но не должно быть пустых ящиков):}
    \smallskip
    
    \[
        \sum\limits^n_{S=k} C^S_n S! C^{k - 1}_{S-1}  
    \]

\section{Формула включения"=исключения.}    

\textbf{Теорема (формула включений"=исключений):}
    \smallskip

    $A = \{A_i\}^n_{i = 1} \;\;\; A_i \subseteq A$
    \bigskip

    $|A \backslash \cup^n_{i = 1} A_i| = |A| - \sum\limits^n_{i = 1} |A_i| +
    \sum\limits_{1 \leq i_1 \leq i_2 \leq n} |A_{i_1} \cap A_{i_2}| - 
    \sum\limits_{1 \leq i_1 \leq i_2 \leq i_3 \leq n} |A_{i_1} \cap A_{i_2} \cap A_{i_3}| +
    \dots + (-1)^k \sum\limits_{1 \leq i_1 \leq \dots \leq i_k \leq n} |A_{i_1} \cap \dots
    \cap A_{i_k}| + (-1)^n |A_1 \cap \dots \cap A_n|.$
    \bigskip

    Доказательство: (ожидается в будущем).

\section{Полиномиальная формула. Свойства полиномиальных коэффициентов.}

\textbf{Теорема (полиномиальная формула):}
    \smallskip

    $\left(\sum\limits^m_{i = 1} x_i\right) = \sum\limits_{k_1 + \dots + k_m = n}
    \overline{P}(k_1, \dots, k_m) x_1^{k_1} x_2^{k_2} \dots x_m^{k_m}$.
    \bigskip

    Доказательство:

    \begin{center}
        \includegraphics[width=150mm]{"13.1.png"}
    \end{center}
    \bigskip

\textbf{Свойства полиномиальных коэффициентов:}
    \smallskip
    
    \begin{enumerate}
        \item{$\sum\limits_{k_1 + \dots + k_m = n} \overline{P}(k_1, \dots, k_m) =
        m^n$;}
        \item{$\overline{P}(k_1 - 1, k_2, \dots, k_m) + 
        \overline{P}(k_1, k_2 - 1, \dots, k_m) + \overline{P}(k_1, k_2, \dots, k_m - 1) =
        \overline{P}(k_1, k_2, \dots, k_m)$
        \bigskip
        
        Доказательство:
        \bigskip
        
        \begin{center}
            \includegraphics[width=150mm]{"13.2.png"}
        \end{center}
        }
    \end{enumerate}

\section{Рекуррентное соотношение $k$"=го порядка, решение рекуррентного соотношения, 
общее решение. Линейные рекуррентные соотношения с постоянными 
коэффициентами. Характеристическое уравнение.}    

\textbf{Определение реккурентного соотношения $k$"=го порядка:}
    \smallskip

    Под реккурентным соотношением $k$"=го порядка понимается формула,
    которая выражает $f(n + k)$ через $F(n + k - 1), f(n + k - 2), \dots,
    f(n)$ предыдущие члены последовательности.
    \bigskip

\textbf{Определение решения реккурентного соотношения:}
    \smallskip
    
    Решением реккурентного соотношения называется числовая последовательность,
    при подстановке общего члена которой в реккурентное соотношение получаем
    верное равенство.
    \bigskip

\textbf{Определение общего решения реккурентного соотношения:}
    \smallskip
    
    Общим решением реккурентного соотношения $k$"=го порядка называется
    решение, зависящее от $k$ постоянных, с помощью которых можно удовлетворить
    любое начальное условие. То есть получить любое общее решение.
    \bigskip

\textbf{Определение линейного реккурентного соотношения с постоянными
коэффициентами:}
    \smallskip
    
    Линейным реккурентным соотношением с постоянными коэффициентами $k$"=го порядка
    называется:
    \smallskip
    
    \[
        f(n + k) = c_1 f(n + k - 1) + c_2 f(n + k - 2) + \dots + c_k f(n)  
    \]
    \smallskip

    Характеристическим уравнением для него называется:
    \smallskip

    \[
        r^k = c_1 r^{k - 1} + c_2 r^{k - 2} + \dots + c_k  
    \]

\section{Линейные реккурентные соотношения с постоянными коэффициентами
второго порядка. Свойства решений.}

\textbf{Определение линейного реккурентного соотношения с постоянными коэффициентами
второго порядка:}
    \smallskip

    \[
        (*) \;\;\; f(n + 2) = c_1 f(n + 1) + c_2 f(n)   
    \]
    \smallskip

    Его характеристическое уравнение имеет вид:
    \smallskip

    \[
       (**) \;\;\; r^2 = c_1 r + c_2  
    \]
    \bigskip

\textbf{Свойства решения линейного реккурентного соотношения с постоянными
коэффициентами второго порядка:}
    \smallskip
    
    \begin{enumerate}
        \item{Если последовательность $\{x_n\}$~--- решение реккурентного соотношения,
        то $\{\alpha x_n\}$ так же является решением;}
        \item{Если $\{x_n\}$ и $\{y_n\}$~--- решения реккурентного соотношения,
        то последовательность $\{x_n + y_n\}$ так же является решением;}
        \item{Если $r_1$~--- это корень (**), то $\{r_1^n\}$~--- решение (*).}
    \end{enumerate}

\section{Решение линейных рекуррентных соотношений с постоянными коэффициентами 
второго порядка в случае равных и различных корней характеристического 
уравнения.}    

\begin{center}
    \includegraphics[width=140mm]{"16.1.png"}
    \bigskip
    
    \includegraphics[width=150mm]{"16.2.png"}
\end{center}

\section{Теорема об общем решении линейных рекуррентных соотношений с постоянными 
коэффициентами $k$"=го порядка. Решение рекуррентных соотношений с 
постоянными коэффициентами $k$"=го порядка с помощью характеристического 
уравнения.}

\textbf{Теорема (общее решение линейного реккурентного соотношения $k$"=го
порядка):}
    \smallskip

    Пусть $(1) \;\;\; f(n + k) = c_1 f(n + k - 1) + \dots + c_k f(n)$~--- линейное реккурентное
    соотношение и $(2) \;\;\; r^k = c_1 r^{k - 1} + \dots + c_k$ его характеристическое
    уравнение. Тогда общее решение (1) можно записать в виде:
    \smallskip

    $f(n) = A_1 + A_2 + \dots + A_p$, где $A_i$ выписывается по действительному
    корню или по паре комплексно сопряжённых корней (2).

    \begin{enumerate}
        \item{Если $x$ дествительный корень (2) кратности $m$, то соответствующее
        ему $A_i$ имеет вид:
        
        \[
            A_i = (c_{i_0} + c_{i_1} n + \dots + c_{i_{m - 1}}n^{m - 1}) x^n;  
        \]}
        \item{Если $r(\cos \varphi \pm i \sin \varphi)$~--- пара комплексно 
        сопряжённых корней (2) кратности 1, то соответствующее им $A_i$ 
        имеет вид:
        
        \[
            A_i = r^n (\cos (n \varphi) D_i + \sin (n \varphi) E_i),  
        \]

        где $D_i$ и $E_i$~--- константы;}
        \item{Если $r(\cos \varphi \pm i \sin \varphi)$~--- пара комплексно
        сопряжённых корней (2) кратности $m$, то соответсвующее им $A_i$
        имеет вид:
        
        \[
            A_i = r^n [ \cos (n \varphi) (D_{i_1} + D_{i_2}n + \dots +
            D_{i_{m - 1}}n^{m - 1}) + 
        \]
        \[    
            + \sin (n \varphi) (E_{i_1} + E_{i_2}n + \dots +
            E_{i_{m - 1}}n^{m - 1})]  
        \]}
    \end{enumerate}
    \bigskip

\textbf{Решение реккурентных соотношений с постоянными коэффициентами $k$"=го порядка
с помощью характеристического уравнения:}
    \smallskip
    
    \begin{center}
        \includegraphics[width=140mm]{"17.1.png"}
        \bigskip

        \includegraphics[width=130mm]{"17.2.png"}
        \bigskip

        \includegraphics[width=130mm]{"17.3.png"}
        \bigskip

        \includegraphics[width=130mm]{"17.4.png"}
        \bigskip

        \includegraphics[width=130mm]{"17.5.png"}
        \bigskip
    \end{center}

\section{Производящая функция. Сумма производящих функций, операция подстановки.}    

\textbf{Определение производящей функции:}
    \smallskip

    Пусть $a_0, a_1, a_2, \dots$ произвольная числовая последовательность. Производящей
    функцией этой последовательности называется выражение вида:

    \[
        a_0 + a_1 t + a_2 t^2 + \dots + a_n t^n + \dots = \sum\limits^{\infty}_{n = 0}
        a_n t^n = A(t)  
    \]
    \bigskip

\textbf{Определение суммы производящих функций:}
    \smallskip

    Пусть имеются производящие функции $A(t) = \sum\limits^{\infty}_{n = 0} a_n t^n$
    и $B(t) = \sum\limits^{\infty}_{n = 0} b_n t^n$. Суммой $A(t)$ и $B(t)$
    называется производящая функция:

    \[
        C(t) = A(t) + B(t) = a_0 + b_0 + (a_1 + b_1)t + (a_2 + b_2)t^2 + \dots
        = \sum\limits^{\infty}_{n = 0} (a_n + b_n) t^n  
    \]
    \bigskip

\textbf{Определение операции подстановки в производящую функцию:}
    \smallskip
    
    Пусть $A(t) = \sum\limits^{\infty}_{n = 0} a_n t^n$ и $B(t) = \sum\limits^{\infty}_{n = 0}
    b_n t^n$ производящие функции, причём $B(0) = b_0 = 0$. Подстановкой в $A(t)$
    $B(t)$ называется производящая функция:

    \[
        C(t) = A(B(t)) = c_0 + c_1 t + c_2 t^2 + \dots =
        a_0 + a_1 (b_1 t + b_2 t^2 + \dots) + a_2 (b_1 t + b_2 t^2 + \dots) +
        \dots,
    \]
    
    где $c_0 = a_0, c_1 = a_1 b_1, c_2 = a_1 b_2 + a_2 b_1^2, \dots$.

\section{Произведение и деление производящих функций.}    

\textbf{Определение произведения производящих функций:}
    \smallskip

    Пусть $A(t) = \sum\limits^{\infty}_{n = 0} a_n t^n$ и $B(t) = \sum\limits^{\infty}_{n = 0}
    b_n t^n$ производящие функции. Произведением $A(t)$ и $B(t)$ будем называть
    производящую функцию:

    \[
        C(t) = A(t) \cdot B(t) = c_0 + c_1 t + c_2 t^2 + \dots,  
    \]

    где $c_0 = a_0 b_0, c_1 = a_0 b_1 + a_1 b_0, c_2 = a_0 b_2 + a_1 b_1 + a_2 b_0,
    \dots, c_n = a_0 b_n + \dots + a_n b_0 = \sum\limits^{n}_{k = 0} a_k b_{n - k}$.
    \bigskip

\textbf{Определение частного производящих функций:}
    \smallskip
    
    Пусть $A(t) = a_0 + a_1 t + \dots$ и $B(t) = b_0 + b_1 t + \dots$ производящие
    функции, причём $B(0) = b_0 \neq 0$. Тогда частным $\frac{A(t)}{B(t)}$
    называется производящая функция:

    \[
        C(t) = \frac{A(t)}{B(t)} = c_0 + c_1 t + \dots,  
    \]

    такая, что $A(t) = B(t) C(t)$. Где $a_0 = b_0 c_0 \Rightarrow c_0 = \frac{a_0}{b_0}$

    $a_1 = b_0 c_1 + b_1 c_0 \Rightarrow c_1 = \frac{a_1 - b_1 c_0}{b_0}$

    $\dots$

    $a_n = b_0 c_n + \dots b_n c_0 \Rightarrow c_n = \frac{a_n - b_1 c_{n - 1} - b_2
    c_{n - 2} - \dots - b_n c_0}{b_0}$

\section{Теорема о разложении функции.}

\textbf{Теорема (о разложении $\frac{1}{(1 - at)^m})$:}
    \smallskip

    \[
        \frac{1}{(1 - at)^m} = 1 + C^1_m a t + C^2_{m + 1} a^2 t^2 + \dots +
        C^n_{m + n - 1} a^n t^n + \dots \;\;\; \forall m \geq 1  
    \]
    \bigskip

    Доказательство (по индукции):
    \bigskip

    \begin{enumerate}
        \item{База: $m = 1$
        \bigskip
        
        $\frac{1}{1 - at} = 1 + at + a^2 t^2 + \dots + a^n t^n + \dots | \cdot (1 + at)$
        \smallskip

        $1 = (1 - at)(1 + at + \dots)$
        \smallskip
        
        $(1 + at + a^2t^2 + a^3t^3 + \dots) - at(1 + at + a^2 t^2 + \dots) = 1$
        \smallskip

        $1 = 1$
        }
        \item{Предположение: $m \geq k$
        \bigskip
        
        $\frac{1}{(1 - at)^k} = 1 + C^1_k a t + C^2_{k + 1} a^2 t^2 + \dots +
        C^n_{k + n - 1} a^n t^n + \dots$
        }
        \item{Шаг индукции: $m \geq k + 1$
        \bigskip
        
        $\frac{1}{(1 - at)^{k + 1}} = 1 + C^1_{k + 1} a t + C^2_{k + 2} a^2 t^2 + 
        \dots + C^n_{k + n} a^n t^n + \dots$
        \smallskip
        
        $\frac{1}{(1 - at)} = (1 - at)\frac{1}{(1 - at)^{k + 1}}$
        \smallskip
        
        $(1 - at)(1 + C^1_{k + 1} at + C^2_{k + 2} a^2t^2 + \dots + C^n_{k + n}
        a^nt^n + \dots) =$
        \smallskip
        
        $ = 1 + (C^1_{k + 1} - 1)at + (C^2_{k + 2} - C^1_{k + 1})
        a^2t^2 + \dots + (C^n_{k + n} - C^{n - 1}_{k + n - 1})a^nt^n + \dots =$
        \smallskip

        $ = 1 + C^1_{k} at + C^2_{k + 1}a^2t^2 + \dots + C^n_{k + n - 1}a^n t^n + \dots$.
        
        }
    \end{enumerate}

\section{Теорема о производящей функции для последовательности, задаваемой линейным 
рекуррентным соотношением. Теорема о рациональной производящей функции.}    

\textbf{Теорема (о производящей функции для последовательности,
заданной реккурентным соотношением):}
    \smallskip

    Пусть последовательность $\{a_n\}$ $a_{n + k} = c_1 a_{n + k - 1} +
    c_2 a_{n + k - 2} + \dots + c_k a_n$ и $a_0, \dots, a_{k - 1}$ заданы.
    Тогда производящая функция для $\{a_n\}$ будет рациональной функцией:

    \[
        A(t) = \frac{P(t)}{Q(t)}  
    \]
    \bigskip

    Доказательство:
    \bigskip

    Пусть $Q(t) = 1 - c_1t - c_2t^2 - \dots - c_kt^k$
    \smallskip

    $P(t) = Q(t) \cdot A(t) = p_0 + p_1t + \dots + p_nt^n + \dots$
    \smallskip

    Так как $A(t) = a_0 + a_1 t + \dots$. Тогда
    \smallskip

    $= a_0 + (a_1 - c_1a_0)t + \dots$
    \smallskip

    $(1 - c_1 t - c_2 t^2 - \dots)(a_0 + a_1 t + a_2 t^2 + \dots)$
    \smallskip

    $p_0 = a_0$
    \smallskip

    $p_1 = a_1 - c_1 a_0$
    \smallskip

    $\dots$
    \smallskip

    $p_{k - 1} = a_{k - 1} - c_1 a_{k - 2} - \dots - c_{k - 1}a_0$
    \smallskip

    $p_k = a_k - c_1 a_{k - 1} - \dots - c_k a_0 = 0$
    \smallskip

    $\dots$
    \smallskip

    $p_{k + n} = a_{n + k} - c_1 a_{n + k - 1} - \dots - c_k a_n = 0$
    \smallskip

    $\dots$
    \bigskip

\textbf{Теорема (о рациональных производящих функциях):}
    \smallskip
    
    Пусть $A(t) = \frac{P(t)}{Q(t)}$ рациональная и $P$ и $Q$ взаимно просты.
    Тогда, начиная с некоторого $n$, последовательность $\{a_n\}$ может
    быть задана линейным реккурентным соотношением $a_{n + k} =
    c_1 a_{n + k - 1} + c_2 a_{n + k - 2} + \dots + c_k a_n$,
    где $c_1, c_2, \dots, c_k$ произвольные константы.

\section{Решение рекуррентных соотношений с помощью производящих функций.}    

\textbf{Алгоритм решения линейных однородных реккурентных соотношений 
с помощью производящих функций:}
    \smallskip

    Пусть $a_{n + k} = c_1 a_{n + k - 1} + \dots + c_k a_n$
    
    \begin{enumerate}
        \item{Выписать $Q(t) = 1 - c_1t - c_2t^2 - \dots - c_kt^k$
        
        $A(t) = a_0 + a_1t + a_2t^2 + \dots$;}
        \item{Найти $P(t)$: $P(t) = Q(t) \cdot A(t);$ ;}
        \item{Разложить $A(t)$ на элементарные дроби:
        
        \[
            A(t) = \frac{P(t)}{Q(t)};  
        \]}
        \item{Воспользоваться теоремой о разложении производящей функции
        и записать её в открытой форме, а так же выписать коэффициент при
        $n$"=ом члене $a_n$.}
    \end{enumerate}

\section{Ориентированные и неориентированные графы.}

\textbf{Определение ориентированного графа:}
    \smallskip

    Ориентированным графом называется:

    \[
        \overrightarrow{G}(V, \rho) \;\;\; \rho \subseteq V \times V,
    \]

    где $V$~--- непустое множество вершин, $\rho$~--- отношение смежности на $V$.
    \smallskip

    Матрица $\rho$ ($M(\rho)$) называется матрицей смежности $\overrightarrow{G}$.

    $(u, \nu) \in \rho$~--- дуга с началом в $u$ и концом в $\nu$.

    Если $|V| = n$, то $M(\rho) = M_{n \times n} = (m_{ij})^n_{i, j = 0}; \;\;\;
    m_{ij} = 
    \begin{cases}
        1, (u_i, \nu_j) \in \rho\\
        0, (u_i, \nu_j) \notin \rho\\
    \end{cases}$
    \bigskip

\textbf{Определение неориентированного графа:}
    \smallskip

    Неориентированным графом называется пара:

    \[
        G = (V, \rho),  
    \]

    где $\rho$~--- симметричное и антирефлексивное отношение на $V$. $\forall
    \{u, \nu\} \in \rho$~--- ребро графа.

\section{Полный граф, дополнение, объединение, соединение графов.}

\textbf{Определение полного графа:}
    \smallskip

    Полным графом называется граф, в котором любые 2 вершины соединены
    ребром.
    \smallskip

    Замечание: степень любой вершины $d(\nu) = n - 1$.
    \bigskip

\textbf{Определение дополнения графа:}
    \smallskip
    
    Граф $\overline{G} = (V^{'}, \rho^{'})$ называется дополнением графа
    $G = (V, \rho)$, если  множества вершин графов $\overline{G}$ и $G$ совпадают,
    то есть $V = V^{'}$,  а множество рёбер $\rho^{'} = V^2 \backslash \rho$.
    Следовательно, любые две вершины, смежные в графе $G$, не смежны в его дополнении
    $\overline{G}$, и любые две вершины не смежные в $G$ смежны в $\overline{G}$.

    \begin{center}
        \includegraphics[width=130mm]{"24.2.png"}
    \end{center}

\textbf{Определение объединения графов:}
    \smallskip
    
    Пусть $G_1 = (V_1, \rho_1)$ $G_2 = (V_2, \rho_2)$

    \[
    G_1 \cup G_2 = (V_1 \cup V_2, \rho_1 \cup \rho_2)
    \]
    \smallskip

    \begin{center}
        \includegraphics[width=130mm]{"24.1.png"}
    \end{center}

\textbf{Определение соединения графов:}    
    \smallskip

    Пусть $G_1 = (V_1, \rho_1)$ $G_2 = (V_2, \rho_2)$ и $V_1 \cap V_2 = \varnothing$

    \[
        G_1 + G_2 = \left[V_1 \cup V_2, (\rho_1 \cup \rho_2) \cup (V_1 \times V_2)
        \cup (V_2 \times V_1) \right]  
    \]
    \smallskip

    \begin{center}
        \includegraphics[width=60mm]{"24.3.png"}
    \end{center}

\section{Теорема о степенном множестве графа.}

\textbf{Теорема (о степенном множестве графа):}
    \smallskip

    Пусть имеется множество натуральных чисел $A = \{d_1, \dots, d_k: k \geq 1$ и $
    d_1 < d_2 < \dots < d_k\}$. Тогда найдётся неориентированный граф $G$ с
    числом вершин $= d_k + 1$, для которого множество $A$ является степенным
    множеством.
    \bigskip

    Доказательство:
    \bigskip

    Пока что не нашёл.

\section{Теорема о соотношении суммы степеней вершин и числа рёбер (лемма
о рукопожатии).}

\textbf{Лемма о рукопожатии:}
    \smallskip

    Для любого графа $G = (V, \rho)$ справедливы утверждения:

    \begin{enumerate}
        \item{$\sum\limits_{\nu \in V} d(\nu) = 2m$, где $m$~--- число рёбер;}
        \item{Количество нечётных вершин чётно;}
        \item{Если в графе $n \geq 2$ вершины, то найдутся по крайней мере две
        вершины с одинаковыми степенями;}
    \end{enumerate}

\section{Алгоритм построения графа по вектору степеней.}

\textbf{Процедура построения изображения графа по вектру степеней:}
    \smallskip

    Пусть есть вектор степеней некоторого графа $(d_1, \dots, d_n)$.

    \begin{enumerate}
        \item{Изобразить $n$ точек с метками $d_1, \dots, d_n$. В качестве
        начальной точки выбрать точку с $d_1$;}
        \item{Начальную точку, с меткой $d_1$, соединить с $d$ точками в порядке
        убывания их меток;}
        \item{Метку начальной точки положить равной $0$, метки всех точек,
        связанных с начальной точкой, уменьшить на 1. Если метки всех точек
        равны 0, то завершаем алгоритм, иначе переходим к шагу 4;}
        \item{В качестве начальной точки выбираем 1 из точек с максимальной меткой.
        Переходим к шагу 2.}
    \end{enumerate}
    \bigskip

    \begin{center}
        \includegraphics[width=150mm]{"27.1.png"}
    \end{center}

\section{Изоморфизм графов. Теорема об изоморфизме графов.}

\textbf{Определение изоморфизма графов:}
    \smallskip

    Будем говорить, что $\overrightarrow{G}_1 = (V_1, \rho_1) \cong 
    \overrightarrow{G_2} = (V_2, \rho_2)$, если существует однозначное
    соответствие $\varphi: V_1 \to V_2$, сохраняющее отношение смежности,
    то есть 
    
    \[
        \forall u \in V_1 \;\;\; \forall \nu \in V_1 \;\;\;\;\;\;
        (u, \nu) \in \rho_1 \Leftrightarrow (\varphi(u), \varphi(\nu)) \in \rho_2
    \]
    \bigskip

\textbf{Теорема (об изоморфизме графов):}
    \smallskip
    
    Пусть $\overrightarrow{G} = (U, \alpha)$ и $\overrightarrow{H} = (V, \beta)$.
    Тогда $\varphi: U \to V$~--- изоморфизм тогда и только тогда, когда
    $A \Phi = \Phi B$. То есть

    \[
        \alpha \cdot \varphi = \varphi \cdot \beta \Leftrightarrow
        \forall u \in U \;\;\; \varphi(\alpha(u)) = \beta (\varphi(u)) \;\;\; (*),  
    \]

    где $A$ и $B$~--- матрицы смежности, $\Phi$~--- матрица изморфизма.
    \bigskip

    Доказательство:
    \bigskip

    \begin{center}
        \includegraphics[width=150mm]{"28.1.png"}
    \end{center}

\section{Проверка на изоморфизм двух графов по их матрицам смежности.}

\begin{enumerate}
    \item{Выписывается матрица $\Phi$ предполагаемого изоморфизма $\varphi$
    как матрица с неопределёнными коэффициентами $\Phi = (\varphi_{ij})$;}
    \item{Составляется матричное уравнение $A \Phi = \Phi B$. Решаем систему,
    находим $\Phi$.}
    \item{Если в каждом столбце и строке ровно одна единица, то изоморфизм есть,
    иначе~--- нет;}
    \item{Если вершина $u^{d^{+}, d^{-}}$ может перейти в вершину $\nu^{d^{+}, d^{-}}$
    (равные степени исхода и захода), то в матрице пишем 1 (если из вершины
    первого орграфа можно лишь единожды попасть в вершину второго орграйфа,
    а если не единожды~--- пишем $\varphi_{ij}$, где $i$~--- номер столбца,
    $j$~--- номер строки), иначе~--- 0.}
\end{enumerate}

\section{Часть графа, подграфы. Путь, цикл, цепь, длина пути и расстояние между ними.
Достаточное условие связности нечётных вершин графа.}

\textbf{Определение части графа:}
    \smallskip

    Частью графа $G = (V, \rho)$ называется граф $G^{'} = (V^{'}, \rho^{'})$, такой,
    что $V^{'} \subseteq V$ и $\rho^{'} \subseteq \rho \cap (V^{'} \times V^{'})$
    \bigskip

    \begin{center}
        \includegraphics[width=40mm]{"30.1.png"}
        \includegraphics[width=40mm]{"30.2.png"}
    \end{center}

    \bigskip

\textbf{Определение подграфа:}
    \smallskip
    
    Подграфом графа $G = (V, \rho)$ называется граф $G^{''} = (V^{''}, \rho^{''})$, 
    такой, что $V^{''} \subseteq V$ и $\rho^{''} = \rho \cap (V^{''} \times V^{''})$.
    \bigskip

    \begin{center}
        \includegraphics[width=40mm]{"30.1.png"}
        \includegraphics[width=40mm]{"30.3.png"}
    \end{center}
    \bigskip

\textbf{Определение пути в графе:}
    \smallskip
    
    Путь~--- последовательность рёбер, в которой два соседних ребра имеют
    общую вершину, и ни одно ребро не встречается более 1 раза.
    \bigskip

\textbf{Определение цикла в графе:}
    \smallskip
    
    Цикл~--- путь, в котором каждая вершина принадлежит не более, чем двум
    рёбрам, и начальная вершина совпадает с конечной.
    \bigskip

\textbf{Определение цепи в графе:}
    \smallskip
    
    Цепь~--- путь, в котором каждая вершина принадлежит не более, чем двум рёбрам.
    \bigskip

\textbf{Определение длины пути в графе:}    
    \smallskip

    Длина пути~--- количество рёбер, входящих в путь.
    \bigskip

\textbf{Определение расстояния между двумя вершинами графа:}
    \smallskip
    
    Расстояние между двумя вершинами~--- длина кратчайшего пути между ними. Если
    пути между вершинами нет, то принято считать расстояние между ними бесконечным.
    \bigskip

\textbf{Достаточное условие связности нечётных вершин графа:}
    \smallskip
    
    Если две нечётные вершины $u$ и $\nu$ в графе~--- единственные нечётные вершины,
    то они связны в графе.
    \bigskip

    Доказательство (от противного):
    \bigskip

    Пусть они лежат в разных компонентах связности, тогда в каждом подграфе
    (компоненте связности) есть лишь одна единственная нечётная вершина~--- 
    противоречие (по лемме о рукопожатии). Получаем, что число нечётных вершин,
    чётно.

\section{Достаточное условие связности графа.}

Если в $n$ вершинном графе число рёбер равно $m > C^2_{n - 1}$, то граф
связный.
\bigskip

Доказательство (от противного):
\bigskip

Пусть $m > C^2_{n - 1}$ и граф не связный. Рассмотрим граф $G = G_1 \ cup G_2$,
где $G_1$~--- произвольная компонента, а $G_2$~--- все вершины из графа,
не находящиеся в $G_1$. Пусть $G_1$ имеет $k$ вершин. Тогда возможны 3 случая:

\begin{enumerate}
    \item{$k = 1$, тогда в $G_2$ находятся $n - 1$ вершин;}
    \item{$k = n - 1$, тогда в $G_2$ находится 1 вершина;}
    \item{$2 \leq  k \leq n - 2$, тогда в $G_2$ находятся $n - k$ вершин.}
\end{enumerate}
\smallskip

Покажем, что все 3 случая противоречивы:
\smallskip

\begin{enumerate}
    \item{В $G_1$ нет рёбер. В $G_2$ может быть до $\frac{(n - 1)(n - 2)}{2}$
    рёбер. Тогда $m \leq \frac{(n - 1)(n - 2)}{2} = C^2_{n - 1}$, следовательно,
    противоречие;}
    \item{Аналогично (1);}
    \item{Оценим $m$:
    \bigskip
    
    \[
        m \leq C^2_k + C^2_{n - k} = \frac{k (k - 1)}{2} + \frac{(n - k)(n - k - 1)}{2}
        = 
    \]
    \[   
        =
        \frac{k^2 - k + n^2 - 2nk + k^2 - n + k}{2} = \frac{2k^2 + n - 2nk - n}{2}.  
    \]
    \smallskip
    
    Рассмотрим разность $C^2_{n - 1} - m$:
    \bigskip
    
    \[
        C^2_{n - 1} - m \geq C^2_{n - 1} - \frac{k^2 - k + n^2 - 2nk + k^2 - n + k}{2} = 
    \]
    \[
        = \frac{(n - 1)(n - 2)}{2} - \frac{k^2 - k + n^2 - 2nk + k^2 - n + k}{2} =
    \]  
    \[
        = \frac{n^2 - 3n + 2 - 2k^2 - n^2 + 2nk + n}{2} = \frac{2nk - 2n + 2 - 2k^2}{2} =  
    \]
    \[
        = n(k - 1) - (k^2 - 1) = (k - 1)(n - (k + 1)) > 0, \;\;\; k \geq 2, n \geq k + 2
        \Rightarrow     
    \]
    \[
        \Rightarrow C^2_{n - 1} - m > 0.  
    \]}
\end{enumerate}

\section{Точка сочленения. Неразделимый граф. Необходимое и достаточное условие
неразделимости связного графа.}

\section{Планарность. Дерево, плоское изображение дерева.}

\end{document}