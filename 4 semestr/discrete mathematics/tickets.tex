\documentclass[14pt]{extarticle}

\usepackage[T2A]{fontenc}
\usepackage[cp1251]{inputenc}
\usepackage[russian]{babel}

\usepackage{graphicx}

\usepackage{amsmath, amsthm, amsfonts, amssymb}
\usepackage[unicode, hidelinks]{hyperref}

\usepackage[
    a4paper, mag=1000,
    left=2.5cm, right=1.5cm, top=2cm, bottom=2cm, bindingoffset=0cm,
    headheight=0cm, footskip=1cm, headsep=0cm
]{geometry}

\usepackage{tempora}

\usepackage{chngcntr} % equation counter manipulations

% reset equation counter at each section
\counterwithin*{equation}{section}
\counterwithin*{equation}{subsection}

\graphicspath{{images/}}

\usepackage{cases}

\begin{document}
    
\tableofcontents

\newpage

\section{Комбинаторика, правило суммы и произведения.
Размещения с повторениями и без повторений.}

\textbf{Правило суммы:}
    \smallskip

    Если объект $A$ можно выбрать $m$ способами, а объект $B$, после выбора $A$, 
    можно выбрать $n$ способами, то пару $(A, B)$ иожно выбрать $n \times m$
    способами.
    \bigskip

\textbf{Правило произведения:}
    \smallskip

    Если $A$ можно выбрать $n$ способами, а $B$~--- $m$ способами, то объект 
    $A$ или $B$ можно выбрать $n + m$ способами. (Выбор $B$ никак не
    согласуется с выбором $A$.)
    \bigskip

\textbf{Размещения с повторениями:}
    \smallskip
    
    Размещениями с повторениями из $n$ типов по $k$ элементов ($k$ и $n$ в произвольном
    соотношении) называются все такие последовательности $k$ элементов,
    принадлижащих $n$ типам, которые отличаются друг от друга составом
    или последовательностью элементов. 
    
    \[
        \overline{A^{k}_n} = n^k 
    \]
    \bigskip

\textbf{Размещения без повторений:}
    \smallskip
    
    Размещениями без повторений из $n$ различных типов по $k$ элементам называются
    все такие последовательности из $k$ различных элементов, такие, что они
    различаются по составу или по порядку. Причём $k < n$.

    \[
        A^k_n = \frac{n!}{(n - k)!}  
    \]

\section{Перестановки с повторениями и без повторений. Сочетания с повторениями 
и без повторений, свойства биномиальных коэффициентов.}

\textbf{Перестановки с повторениями:}
    \smallskip

    Перестановками с повторениями из $n_1, \dots, n_k$ элементов $k$"=го типа 
    называются всевозможные последовательности длины $n$, отличающиеся друг от
    друга последовательностью элементов.

    \[
        \overline{P}(n_1, \dots, n_k) = \frac{n!}{n_1! \cdot \dots \cdot n_k!}  
    \]
    \bigskip

\textbf{Перестановски без повторений:}
    \smallskip
    
    Перестановками без повторений из $n$ элементов называются всевозможные
    последовательности из $n$ элементов.

    \[
        P_n = n!  
    \]
    \bigskip

\textbf{Сочетания с повторениями:}
    \smallskip
    
    Сочетаниями с повторениями из $n$ по $k$ ($k$ и $n$ в произвольном соотношении)
    называются все такие комбинации из $k$ элементов $\in n$ типам,
    которые отличаются только составом элементов.

    \[
        \overline{C^k}_n = C^k_{n + k - 1} = \overline{P}(n - 1, k) 
    \]
    \bigskip

\textbf{Сочетания без повторений:}
    \smallskip
    
    Сочетаниями без повторений из $n$ по $k$ ($k \leq n$) называются
    все такие комбинации из $k$ различных элементов, выбранных из
    $n$ исходных элементов, которые отличаются друг от друга составом.

    \[
        C^k_n = \frac{n!}{(n - k)! k!}  
    \]
    \bigskip

\textbf{Свойства биномиальных коэффициентов:}
    \smallskip
    
    \begin{enumerate}
        \item{
            \[
                C^k_n = \overline{P}(k, n - k)  
            \]
        }
        \item{
            \[
                C^k_n = C^{n - k}_n  
            \]
        }
        \item{
            \[
                C^k_n = C^{k - 1}_{n - 1} + C^{k}_{n - 1}  
            \]
        }
        \item{
            \[
                C^0_n + C^1_n + \dots + C^n_n = 2^n  
            \]
        }
        \item{
            \[
                C^0_n - C^1_n + C^2_n - C^3_n + \dots + C^n_n = 0  
            \]
        }
    \end{enumerate}

\end{document}