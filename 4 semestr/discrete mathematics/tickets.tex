\documentclass[14pt]{extarticle}

\usepackage[T2A]{fontenc}
\usepackage[cp1251]{inputenc}
\usepackage[russian]{babel}

\usepackage{graphicx}

\usepackage{amsmath, amsthm, amsfonts, amssymb}
\usepackage[unicode, hidelinks]{hyperref}

\usepackage[
    a4paper, mag=1000,
    left=2.5cm, right=1.5cm, top=2cm, bottom=2cm, bindingoffset=0cm,
    headheight=0cm, footskip=1cm, headsep=0cm
]{geometry}

\usepackage{tempora}

\usepackage{chngcntr} % equation counter manipulations

% reset equation counter at each section
\counterwithin*{equation}{section}

\graphicspath{{images/}}

\begin{document}
    
\tableofcontents

\newpage

\section{Комбинаторика, правило суммы и произведения.
Размещения с повторениями и без повторений.}

\textbf{Правило суммы:}
    \smallskip

    Если объект $A$ можно выбрать $m$ способами, а объект $B$, после выбора $A$, 
    можно выбрать $n$ способами, то пару $(A, B)$ можно выбрать $n \times m$
    способами.
    \bigskip

\textbf{Правило произведения:}
    \smallskip

    Если $A$ можно выбрать $n$ способами, а $B$~--- $m$ способами, то объект 
    $A$ или $B$ можно выбрать $n + m$ способами. (Выбор $B$ никак не
    согласуется с выбором $A$.)
    \bigskip

\textbf{Размещения с повторениями:}
    \smallskip
    
    Размещениями с повторениями из $n$ типов по $k$ элементов ($k$ и $n$ в произвольном
    соотношении) называются все такие последовательности $k$ элементов,
    принадлижащих $n$ типам, которые отличаются друг от друга составом
    или последовательностью элементов. 
    
    \[
        \overline{A^{k}_n} = n^k 
    \]
    \bigskip

\textbf{Размещения без повторений:}
    \smallskip
    
    Размещениями без повторений из $n$ различных типов по $k$ элементам называются
    все такие последовательности из $k$ различных элементов, такие, что они
    различаются по составу или по порядку. Причём $k < n$.

    \[
        A^k_n = \frac{n!}{(n - k)!}  
    \]

\section{Перестановки с повторениями и без повторений. Сочетания с повторениями 
и без повторений, свойства биномиальных коэффициентов.}

\textbf{Перестановки с повторениями:}
    \smallskip

    Перестановками с повторениями из $n_1, \dots, n_k$ элементов $k$"=го типа 
    называются всевозможные последовательности длины $n$, отличающиеся друг от
    друга последовательностью элементов.

    \[
        \overline{P}(n_1, \dots, n_k) = \frac{n!}{n_1! \cdot \dots \cdot n_k!}  
    \]
    \bigskip

\textbf{Перестановски без повторений:}
    \smallskip
    
    Перестановками без повторений из $n$ элементов называются всевозможные
    последовательности из $n$ элементов.

    \[
        P_n = n!  
    \]
    \bigskip

\textbf{Сочетания с повторениями:}
    \smallskip
    
    Сочетаниями с повторениями из $n$ по $k$ ($k$ и $n$ в произвольном соотношении)
    называются все такие комбинации из $k$ элементов $\in n$ типам,
    которые отличаются только составом элементов.

    \[
        \overline{C^k}_n = C^k_{n + k - 1} = \overline{P}(n - 1, k) 
    \]
    \bigskip

\textbf{Сочетания без повторений:}
    \smallskip
    
    Сочетаниями без повторений из $n$ по $k$ ($k \leq n$) называются
    все такие комбинации из $k$ различных элементов, выбранных из
    $n$ исходных элементов, которые отличаются друг от друга составом.

    \[
        C^k_n = \frac{n!}{(n - k)! k!}  
    \]
    \bigskip

\textbf{Свойства биномиальных коэффициентов:}
    \smallskip
    
    \begin{enumerate}
        \item{
            \[
                C^k_n = \overline{P}(k, n - k)  
            \]
        }
        \item{
            \[
                C^k_n = C^{n - k}_n  
            \]
        }
        \item{
            \[
                C^k_n = C^{k - 1}_{n - 1} + C^{k}_{n - 1}  
            \]
        }
        \item{
            \[
                C^0_n + C^1_n + \dots + C^n_n = 2^n  
            \]
        }
        \item{
            \[
                C^0_n - C^1_n + C^2_n - C^3_n + \dots + C^n_n = 0  
            \]
        }
    \end{enumerate}

\section{Сколькими способами можно разложить $n_1$ предметов одного сорта,
$\dots, n_k$ предметов $k$"=го сорта в два ящика? Следствия.}

\textbf{Схема: $n_1$ предметов $1$"=го типа $\dots$ $n_k$ предметов $k$"=го типа
раскладываются в два различных ящика:}
    \smallskip

    \begin{center}
        $(n_1 + 1) \cdot (n_2 + 1) \cdot \dots \cdot (n_k + 1)$ способов.
    \end{center}
    \bigskip
 
\textbf{Следствие 1:}
    \smallskip
    
    Если все предметы различны, то:

    \begin{center}
        $n_1 = 1 = n_2 = \dots = n_k = 1 \Rightarrow 2^k$ способов.
    \end{center}
    \bigskip

\textbf{Следствие 2:}
    \smallskip
    
    Не менее $r_i$ предметов $i$"=го типа в каждый ящик:

    \begin{center}
        $(n_1 - 2r_1 + 1) \cdot (n_2 - 2r_2 + 1) \cdot \dots 
        \cdot (n_k - 2r_k + 1)$ способов.
    \end{center}

\section{Даны $n$ различных предметов и $k$ ящиков. Требуется положить в первый
ящик $n_1$ предметов, в $k$"=ый~--- $n_k$ предметов, где $n_1 + \dots + n_k = n$.
Сколькими способами можно сделать такое распределение, если не интересует порядок
распределения предметов в ящике?}    

\textbf{Схема: $n$ различных предметов раскладываются в $k$ различных
ящиков (порядок внутри ящиков не важен):}
    \smallskip

    \begin{center}
        $\frac{n!}{n_1! \cdot \dots \cdot n_k!}$ способов
    \end{center}

\section{Даны $n$ различных предметов и $k$ одинаковых ящиков. Требуется положить в 
каждый ящик $n = \frac{n}{k}$ предметов. Сколькими способами можно сделать такое 
распределение, если не интересует порядок предметов в ящике и все ящики 
одинаковы?}    

\textbf{Схема: $n$ различных предметов в $k$ одинаковых ящиков
(порядок внутри ящиков не важен) $\frac{n}{k}$ предметов в каждый ящик:}
    \smallskip

    \begin{center}
        $\frac{n!}{k!((\frac{n}{k})!)^{k}}$ способов.
    \end{center}

\section{Сколькими способами можно распределить $n$ одинаковых предметов в $k$ ящиков?}    

\textbf{Схема: $n$ одинаковых предметов в $k$ разных ящиков:}
    \smallskip

    \begin{center}
        $\overline{P}(n, k-1) = \frac{(n + k - 1)!}{n!(k - 1)!}$ способов.
    \end{center}

\section{Сколько существует способов разложить n различных  предметов в k ящиков, если 
нет никаких ограничений?}    

\textbf{Схема: $n$ различных предметов в $k$ разных ящиков:}
    \smallskip

\begin{center}
    $k^n$ способов.
\end{center}

\section{Сколькими способами можно положить  n различных предметов в k ящиков, если 
не должно быть пустых ящиков?}

\textbf{Схема: $n$ различных предметов в $k$ разных ящиков, причём не должно
быть пустых ящиков:}
    \smallskip

\begin{center}
    $A_i$~--- количество способов, когда $i$="ый ящик пустой.
    \bigskip

    $|A \backslash \cup^{k}_{i = 1} A_i| = |A| - \sum^{k}_{i = 1} |A_i| + \sum |A_i \cap A_j| +
    \dots + (-1)^{k - 1} \sum_{i_1 \dots i_{k - 1}} |A_{i_1} \cap \dots \cap A_{i_{k - 1}}| +
    (-1)^{k} |A_1 \cap \dots \cap A_k| = k^n - C^1_k (k - 1)^n + C^2_k(k - 2)^n +
    \dots + (-1)^{k - 1} C^{k - 1}_k \cdot 1^n$ способов.
\end{center}

\section{Имеется $n_1$ предметов одного сорта$, \dots, n_s$~---$s$"=го сорта.
Сколькими способами их можно разложить по $k$ ящикам, если не должно быть
пустых ящиков?}

\textbf{Схема $n_1$ предметов первого типа $\dots n_m$ предметов $m$"=го типа
по $k$ различным ящикам, причём нет пустых ящиков:}
    \smallskip

    $A_i$~--- $i$"=ый ящик пустой $i = \overline{1, k}$.
    \bigskip

    $|A| = C^{k - 1}_{n_1 + k - 1} \cdot C^{k - 1}_{n_2 + k - 1} \cdot \dots
    \cdot C^{k - 1}_{n_m + k - 1}$
    \bigskip

    $|A_i| = C^{k - 2}_{n_1 + k - 2} \cdot C^{k - 2}_{n_2 + k - 2} \cdot \dots
    \cdot C^{k - 2}_{n_m + k - 2}$
    \bigskip

    $|A \backslash \cup^{k}_{i = 1} A_i| = C^{k - 1}_{n_1 + k - 1} \cdot \dots
    \cdot C^{k - 1}_{n_m + k - 1} - C^1_{k} C^{k - 2}_{n_1 + k - 2} \cdot \dots
    \cdot C^{k - 2}_{n_m + k - 2} + C^2_k C^{k - 3}_{n_1 + k - 3} \cdot \dots
    \cdot C^{k - 3}_{n_m + k - 3} + \dots + (-1)^{k - 1} C^{k - 1}_{k} 1^n$.

\section{Сколько существует способов разложить $n$ различных  предметов в $k$  различных 
ящиков с учетом расположения предметов в ящиках, если все $n$ предметов должны 
быть использованы? Следствие.}

\textbf{Схема: $n$ различных предметов в $k$ различных ящиков (порядок внутри
ящиков важен, ящики могут быть пустыми):}
    \smallskip

    \begin{center}
        $\overline{P}(i_1, \dots, i_m, k-1) = \frac{(n + k - 1)!}{(k - 1)!} =
        A^n_{n + k - 1}$,
    \end{center}

    где $i_1 + \dots + i_m = n$. $i_j$~--- количество предмотов в ящике под
    номером $j$.
    \bigskip

\textbf{Следствие (та же схема, но не должно быть пустых ящиков):}
    \smallskip    
    
    \[
        A^k_n A^{n-k}_{n-k+k-1} = A^k_n A^{n-k}_{n-1} = \frac{n!}{(n-k)!} =
        = \frac{(n-1)!}{(k-1)!} = n! C^{k-1}_{n-1}    
    \]

\section{Сколько существует способов разложить $n$ различных  предметов в $k$  различных 
ящиков с учетом расположения предметов в ящиках, если не все $n$ предметов могут 
быть использованы и некоторые ящики могут оказаться пустыми? Следствие.}    

\textbf{Схема: $n$ различных предметов в $k$ различных ящиков (порядок
внутри ящиков важен, можно использовать не все предметы):}
    \smallskip

    $S$~--- в распределении учавствует $s$ предметов. $S = \overline{0, n}$.
    \bigskip

    \[
        \sum\limits^n_{S=0} C^S_n A^S_{S + k - 1}  
    \]
    \bigskip

\textbf{Следствие (та же схема, но не должно быть пустых ящиков):}
    \smallskip
    
    \[
        \sum\limits^n_{S=k} C^S_n S! C^{k - 1}_{S-1}  
    \]

\section{Формула включения"=исключения.}    

\textbf{Теорема (формула включений"=исключений):}
    \smallskip

    $A = \{A_i\}^n_{i = 1} \;\;\; A_i \subseteq A$
    \bigskip

    $|A \backslash \cup^n_{i = 1} A_i| = |A| - \sum\limits^n_{i = 1} |A_i| +
    \sum\limits_{1 \leq i_1 \leq i_2 \leq n} |A_{i_1} \cap A_{i_2}| - 
    \sum\limits_{1 \leq i_1 \leq i_2 \leq i_3 \leq n} |A_{i_1} \cap A_{i_2} \cap A_{i_3}| +
    \dots + (-1)^k \sum\limits_{1 \leq i_1 \leq \dots \leq i_k \leq n} |A_{i_1} \cap \dots
    \cap A_{i_k}| + (-1)^n |A_1 \cap \dots \cap A_n|.$
    \bigskip

    Доказательство: (ожидается в будущем).

\section{Полиномиальная формула. Свойства полиномиальных коэффициентов.}

\textbf{Теорема (полиномиальная формула):}
    \smallskip

    $\left(\sum\limits^m_{i = 1} x_i\right) = \sum\limits_{k_1 + \dots + k_m = n}
    \overline{P}(k_1, \dots, k_m) x_1^{k_1} x_2^{k_2} \dots x_m^{k_m}$.
    \bigskip

    Доказательство:

    \begin{center}
        \includegraphics[width=150mm]{"13.1.png"}
    \end{center}
    \bigskip

\textbf{Свойства полиномиальных коэффициентов:}
    \smallskip
    
    \begin{enumerate}
        \item{$\sum\limits_{k_1 + \dots + k_m = n} \overline{P}(k_1, \dots, k_m) =
        m^n$;}
        \item{$\overline{P}(k_1 - 1, k_2, \dots, k_m) + 
        \overline{P}(k_1, k_2 - 1, \dots, k_m) + \overline{P}(k_1, k_2, \dots, k_m - 1) =
        \overline{P}(k_1, k_2, \dots, k_m)$
        \bigskip
        
        Доказательство:
        \bigskip
        
        \begin{center}
            \includegraphics[width=150mm]{"13.2.png"}
        \end{center}
        }
    \end{enumerate}

\section{Рекуррентное соотношение $k$"=го порядка, решение рекуррентного соотношения, 
общее решение. Линейные рекуррентные соотношения с постоянными 
коэффициентами. Характеристическое уравнение.}    

\textbf{Определение реккурентного соотношения $k$"=го порядка:}
    \smallskip

    Под реккурентным соотношением $k$"=го порядка понимается формула,
    которая выражает $f(n + k)$ через $F(n + k - 1), f(n + k - 2), \dots,
    f(n)$ предыдущие члены последовательности.
    \bigskip

\textbf{Определение решения реккурентного соотношения:}
    \smallskip
    
    Решением реккурентного соотношения называется числовая последовательность,
    при подстановке общего члена которой в реккурентное соотношение получаем
    верное равенство.
    \bigskip

\textbf{Определение общего решения реккурентного соотношения:}
    \smallskip
    
    Общим решением реккурентного соотношения $k$"=го порядка называется
    решение, зависящее от $k$ постоянных, с помощью которых можно удовлетворить
    любое начальное условие. То есть получить любое общее решение.
    \bigskip

\textbf{Определение линейного реккурентного соотношения с постоянными
коэффициентами:}
    \smallskip
    
    Линейным реккурентным соотношением с постоянными коэффициентами $k$"=го порядка
    называется:
    \smallskip
    
    \[
        f(n + k) = c_1 f(n + k - 1) + c_2 f(n + k - 2) + \dots + c_k f(n)  
    \]
    \smallskip

    Характеристическим уравнением для него называется:
    \smallskip

    \[
        r^k = c_1 r^{k - 1} + c_2 r^{k - 2} + \dots + c_k  
    \]

\section{Линейные реккурентные соотношения с постоянными коэффициентами
второго порядка. Свойства решений.}

\textbf{Определение линейного реккурентного соотношения с постоянными коэффициентами
второго порядка:}
    \smallskip

    \[
        (*) \;\;\; f(n + 2) = c_1 f(n + 1) + c_2 f(n)   
    \]
    \smallskip

    Его характеристическое уравнение имеет вид:
    \smallskip

    \[
       (**) \;\;\; r^2 = c_1 r + c_2  
    \]
    \bigskip

\textbf{Свойства решения линейного реккурентного соотношения с постоянными
коэффициентами второго порядка:}
    \smallskip
    
    \begin{enumerate}
        \item{Если последовательность $\{x_n\}$~--- решение реккурентного соотношения,
        то $\{\alpha x_n\}$ так же является решением;}
        \item{Если $\{x_n\}$ и $\{y_n\}$~--- решения реккурентного соотношения,
        то последовательность $\{x_n + y_n\}$ так же является решением;}
        \item{Если $r_1$~--- это корень (**), то $\{r_1^n\}$~--- решение (*).}
    \end{enumerate}

\section{Решение линейных рекуррентных соотношений с постоянными коэффициентами 
второго порядка в случае равных и различных корней характеристического 
уравнения.}    

\begin{center}
    \includegraphics[width=140mm]{"16.1.png"}
    \bigskip
    
    \includegraphics[width=150mm]{"16.2.png"}
\end{center}

\end{document}