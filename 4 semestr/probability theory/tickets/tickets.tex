\documentclass[14pt]{extarticle}

\usepackage[T2A]{fontenc}
\usepackage[cp1251]{inputenc}
\usepackage[russian]{babel}

\usepackage{graphicx}

\usepackage{amsmath, amsthm, amsfonts, amssymb}
\usepackage[unicode, hidelinks]{hyperref}

\usepackage[
    a4paper, mag=1000,
    left=2.5cm, right=1.5cm, top=2cm, bottom=2cm, bindingoffset=0cm,
    headheight=0cm, footskip=1cm, headsep=0cm
]{geometry}

\usepackage{tempora}

\usepackage{chngcntr} % equation counter manipulations

% reset equation counter at each section
\counterwithin*{equation}{section}
\counterwithin*{equation}{subsection}

\graphicspath{{images/}}

\usepackage{cases}

\begin{document}

\tableofcontents

\newpage

\section{Мнимум 1}

\subsection{Глава 1}

\subsubsection{Билет 1}

\textbf{Определение случайного события:}
    \smallskip

    Пусть $\Omega$ "--- множество элементарных
    исходов эксперимента. Случайным событием
    называется любое подмножество множества
    $\Omega$.
    \bigskip

\textbf{Определение достоверного события:}
    \smallskip

    Достоверным событием называется событие
    $\Omega$, которому благоприятствует
    каждый исход эксперимента.
    \bigskip

\textbf{Определение невозможного события:}
    \smallskip
    
    Невозможным событием называется пустое
    множество, которому не благоприятствует ни
    один исход эксперимента.
    \bigskip

\textbf{Определение противоположного события:}
    \smallskip
    
    Событием, противоположным событию 
    $A$ называется событие $\overline{A}$, которое
    состоит из исходов, не благоприятствующих $A$.
    \bigskip

\subsubsection{Билет 2}

\textbf{Определение $\sigma$"=алгебры событий:}
    \smallskip

    Сигма алгеброй событий называется множество
    $\mathcal{F}$ подмножеств $A \subset \Omega$,
    удовлетворяющее условиям:

    \begin{enumerate}
        \item{если $A \in \mathcal{F}$, то
        $\overline{A} \in \mathcal{F}$;}
        \item{$\Omega \in \mathcal{F}$;}
        \item{если $\{A\}_{i = 1}^{\infty} \in
        \mathcal{F}$, то
        $\bigcup\limits_{i = 1}^{\infty} A_i \in 
        \mathcal{F}$.}
    \end{enumerate}
    \bigskip

\subsubsection{Билет 3}

Пусть $\mathcal{A}$ "--- алгебра множеств из
$\Omega$.

\textbf{Определение конечно аддитивной
вероятностной меры}
    \smallskip

    Конечно аддитивной вероятностной мерой $Q(A)$
    называется функция множества $Q: \mathcal{A}
    \rightarrow [0; 1]$, такая, что:

    \begin{enumerate}
        \item{$\forall A \in \mathcal{A} \;\;\;
        Q(A) \geq 0$;}
        \item{$Q(\Omega) = 1$;}
        \item{$\forall A, B \in \mathcal{A} : 
        \;\;\; A \cap B = \varnothing \;\;\;
        Q(A \cup B) = Q(A) + Q(B) \;\;\;
        Q\left(\bigcup\limits_{i = 1}^{\infty} A_i\right) = \sum_{i = 1}^{\infty}
        Q(A_i)$.}
    \end{enumerate}
    \bigskip  

\textbf{Определение счётно аддитивной 
вероятностной меры:}
    \smallskip

    Счётно аддитивно вероятностной мерой $P(A)$
    называется функция множества $P: \mathcal{F}
    \rightarrow [0; 1]$, такая, что:

    \begin{enumerate}
        \item{$\forall A \in \mathcal{F} \;\;\;
        P(A) \geq 0$;}
        \item{$P(\Omega) = 1$;}
        \item{$\forall \{A_i\}_{i = 1}^{\infty}
        \in \mathcal{F} : \;\;\; \forall
        i \neq j \;\;\; A_i \bigcap A_j = 
        \varnothing \;\;\; P\left(\bigcup\limits_{i = 1}^{
        \infty} A_i\right) = \sum_{i = 1}^{\infty}
        P(A_i)$.}
    \end{enumerate}
    \bigskip    

\subsubsection{Билет 4:}

\textbf{Свойства вероятности:}
    \smallskip

    \begin{enumerate}
        \item{$P(\overline{A}) = 1 - P(A)$}
        \item{Если $A \subseteq B$, то 
        $ P(A) \leq P(B)$ и $P(B \setminus A) = 
        P(B) - P(A)$}
        \item{
            \textbf{Теория сложения вероятностей:}
                \smallskip
                
                Пусть $A$ и $B$ некоторые события,
                $A, B \in \mathcal{F}$. Тогда 

                \[
                    P(A \cup B) = P(A) + P(B) -
                    P(A \cap B)
                \]
        }
        \item{
            \textbf{Непрерывность вероятностной
            меры:}
                \smallskip

                Пусть $\{A\}_{i = 1}^{\infty}$ "---
                монотонный класс событий, то есть

                1) $A_i \subset A_{i + 1}$ или
                2) $A_i \supset A_{i + 1}$. Тогда

                \[
                    P(\lim\limits_{n \to \infty} (A_n)) =
                    \lim\limits_{n \to \infty} (A_n)
                \]
        }
    \end{enumerate}
    \bigskip

\subsubsection{Билет 5}

\textbf{Определение классической вероятности:}
    \smallskip

    $P(A) = \frac{k}{n}$, где k "--- количество
    благоприятных $A$ исходов, n "--- количество
    всех возможных исходов эксперимента.
    \bigskip

\subsubsection{Билет 6}

\textbf{Определение условной вероятности:}
    \smallskip

    Пусть $(\Omega, \mathcal{F}, P)$ "--- 
    вероятностное пространство и $A, B \in
    \mathcal{F}, \;\; P(B) > 0$. Условной
    вероятностью события $A$ при условии, что
    наступило событие $B$ называется число:

    \[
        P(A|B) = \frac{P(A \cap B)}{P(B)}    
    \]
    \bigskip

Пусть $\{A_i\}_{i = 1}^{\infty}$ "--- полная
группа несовменстных событий.

Назовём события $A_i$ гипотезами, а $P(A_i)$
назовём априорные вероятности гипотез.
\bigskip

\textbf{Теорема (формула полной вероятности):}
    \smallskip

    Пусть $(\Omega, \mathcal{F}, P)$ "--- 
    вероятностное пространство и
    $\{A_i\}_i^{\infty} \in \mathcal{F}$ "---
    полная группа попарно несовметных событий;
    $P(A_i) > 0$, пусть $A \in \mathcal{F}$ "---
    неполное событие и $P(A|A_i) \geq 0$. Тогда

    \[
        P(A) = \sum_{i = 1}^{\infty} P(A_i) *
        P(A|A_i)    
    \]
    \bigskip

\textbf{Теорема (формула Байеса):}
    \smallskip
    
    Пусть $\{A_i\}_{i = 1}^{\infty}$ "--- полная
    группа попарно несовместимых событий, и 
    пусть для некоторого $P(A) > 0$. Тогда

    \[
        \forall i = \overline{1, \infty} \;\;
        P(A_i|A) = \frac{P(A_i)\cdot P(A|A_i)}{P(A)},  
    \]
    \bigskip

\subsubsection{Билет 7}

\textbf{Определение независимости событий:}
    \smallskip

    Случайные события $A$ и $B$ называются 
    независимыми, если 

    \[
        P(A \cap B) = P(A)\cdot P(B)  
    \]
    \bigskip

\subsubsection{Билет 8}

\textbf{Определение (критерий независимости
событий):}
    \smallskip

    Пусть $A$ и $B$ такие, что $P(B) > 0$.
    Тогда случайные события $A$ и $B$ независимы
    $\Leftrightarrow  P(A|B) = P(A)$
    \bigskip

\textbf{Теорема (о независимости противоположных
событий):}
    \smallskip
    
    Пусть $A$ и $B$ "--- независимы. Тогда события
    $A$ и $\overline{B}$; $\overline{A}$ и $B$;
    $\overline{A}$ и $\overline{B}$ "--- попарно
    независимы.
    \bigskip

\subsection{Глава 2}

\subsubsection{Билет 1}

Множество $A$ называется элементарным, если 
оно представимо в виде суммы прямоугольников
хотя бы 1 способом:

\[
    A = \bigcup P_k,
\]

где ${P_k}$ "--- покрытие.
\bigskip


Мерой элементарного множества $A$ называется

\[
    m^{'}(A) = \sum m (P_k),
\]

где ${P_k}$ "--- разбиение $A$.
\bigskip

\textbf{Определение верхней меры Лебега:}
    \smallskip

    Верхней мерой Лебега называется

    \[
        \mu^{*}(A) = \inf_{\{P_k\}} \sum_m (P_k)  
    \]
    \bigskip

\textbf{Определение нижней меры Лебега:}
    \smallskip

    Рассмотрим множество $E \backslash A. \;\;\;
    (m(E) = 1).$ 

    Нижней мерой Лебега называется

    \[
        \mu_{*}(A) = 1 - \mu^{*}(E \setminus A)
    \]
    \bigskip

\textbf{Определение измеримого по Лебегу 
множества и меры Лебега:}
    \smallskip
    
    Говорят, что множество $A$ измеримо по Лебегу,
    если:

    \[
        \mu^{*}(A) = \mu_{*}(A) = \mu (A)
    \]

    Величина $\mu(A)$ "--- мера Лебега множества
    $A$.
    \bigskip

\subsubsection{Билет 2}

\textbf{Определение измеримой функции:}
    \smallskip

    Пусть $X$ и $Y$ "--- некоторые множества
    и пусть $S_x$ и $S_y$ "--- классы подмножества.
    $f: X \rightarrow Y$ "--- некоторая функция.
    \smallskip

    Функция $f: X \rightarrow Y$ называется 
    $(S_x, S_y)$ "--- измеримой, если:

    \[
        \forall B \in S_y \;\;\;\; \exists f^{-1}
        (B) \in S_x
    \]
    \bigskip

\textbf{Определение измеримой действительной
функции:}
    \smallskip
    
    Действительная функция $f(x)$ с областью 
    определения $X \subset R$ называется 
    $\mu$"=измеримой или $S_{\mu}$"=измеримой,
    если для любого борелевского множества $b \in
    \beta(R) \;\;\;\; f^{-1}(b) \in S_{\mu}$.
    \bigskip

\subsubsection{Билет 3}

\textbf{Определение (критерий измеримости 
действительных функций):}
    \smallskip

    Действительная функция $f(x)$ измерима
    $\Leftrightarrow$

    \begin{center}

        $
            \forall C \in R \;\;\;\; f^{-1}(b) = 
            f^{-1}(-\infty, C)
        $ "--- измерима

        или 

        $\{x : f(x) < C\}$

    \end{center}
    \bigskip

\subsubsection{Билет 4}

\textbf{Определение случайной величины:}
    \smallskip

    Пусть $(\Omega, \mathcal{F}, P)$ "---
    вероятностное пространство. Случайной 
    величиной называется вещественно значная
    функция $\xi$ такая, что

    \[
        \xi : \Omega \rightarrow R \;\;\;\;
        \forall x \in R \;\;\;\; \{w: \xi(w) < x\}
        \in \mathcal{F}
    \]
    \bigskip

\subsubsection{Билет 5}

\textbf{Определение функции распределения:}
    \smallskip

    Функцией распределения вероятностей 
    случайной величины $\xi$ называется функция 
    $F_{\xi}(x) = P\{w : \xi (w) < x\}$
    \bigskip

\textbf{Свойства функции распределения:}
    \smallskip
    
    \begin{enumerate}
        \item{$0 \leq F_{\xi}(x) \leq 1 \;\;\;\;
        \forall x \in R$;}
        \item{$F_{\xi}(x)$ "--- неубывающая,
        непрерывная слева функция;}
        \item{$\lim\limits_{x \to +\infty} F_{\xi} (x) = 1$
        
        $\lim\limits_{x \to -\infty} F_{\xi} (x) = 0$;}
        \item{$P\{a \leq \xi < b\} = F_{\xi}(b) -
        F_{\xi}(a)$;}
        \item{$P\{\xi = x_0\} = F_{\xi}(x_0 + 0)
        - F_{\xi}(x_0)$.}
    \end{enumerate}
    \bigskip

\subsubsection{Билет 6}

\textbf{Определение функции плотности
распределения:}
    \smallskip

    Функцией плотности распределения вероятностей
    случайной величины $\xi$ называется функция
    $f_{\xi}(x)$ такая, что:

    \begin{enumerate}
        \item{$\forall x \in R \;\;\;\; f_{\xi}
        (x) \geq 0$;}
        \item{$F_{\xi}^{'}(x) = f_{\xi}(x)$;}
        \item{$\int\limits_{-\infty}^{\infty} f_{\xi}(x)
        dx = 1$;}
        \item{$P\{a \leq \xi \leq b\} = \int\limits_b^{a}
        f_{\xi}(x) dx$.}
    \end{enumerate}
    \bigskip

\subsubsection{Билет 7}

\textbf{Определение случайных независимых
велечин:}
    \smallskip

    Случайные велечины $\xi$ и $\mu$ называются
    независимыми, если:

    \[ 
        \forall x, y \in R \;\;\;\; P\{w : \xi(w)
        < x; \mu (w) < y\} = P\{w: \xi (w) < x\} \cdot
        P\{w : \mu (w) < y\}
    \]

\section{Минимум 2}

\subsection{Билет 1}

\textbf{Определение случайной величины:}
\smallskip

Пусть $(\Omega, \mathcal{F}, P)$ "---
вероятностное пространство. Случайной 
величиной называется вещественно значная
функция $\xi$ такая, что

\[
    \xi : \Omega \rightarrow R \;\;\;\;
    \forall x \in R \;\;\;\; \{w: \xi(w) < x\}
    \in \mathcal{F}
\]
\bigskip

\subsection{Билет 2}

\textbf{Определение функции распределения:}
\smallskip

Функцией распределения вероятностей 
случайной величины $\xi$ называется функция 
$F_{\xi}(x) = P\{w : \xi (w) < x\}$
\bigskip

\textbf{Свойства функции распределения:}
\smallskip

\begin{enumerate}
    \item{$0 \leq F_{\xi}(x) \leq 1 \;\;\;\;
    \forall x \in R$;}
    \item{$F_{\xi}(x)$ "--- неубывающая,
    непрерывная слева функция;}
    \item{$\lim\limits_{x \to +\infty} F_{\xi} (x) = 1$
    
    $\lim\limits_{x \to -\infty} F_{\xi} (x) = 0$;}
    \item{$P\{a \leq \xi < b\} = F_{\xi}(b) -
    F_{\xi}(a)$;}
    \item{$P\{\xi = x_0\} = F_{\xi}(x_0 + 0)
    - F_{\xi}(x_0)$.}
\end{enumerate}
\bigskip  

\subsection{Билет 3}

\textbf{Доказательство непрерывности слева функции 
распределения}
\smallskip

Требуется показать, что для возрастающей последовательности
$\{x_n\}$, такой что $\lim\limits_{n \to \infty} x_n = x$,
последовательность $\{F(x_n)\}$ при $n \to \infty$ стремится
к $F(x)$ или $\lim\limits_{n \to \infty} F(x_n) = F(x)$.
\smallskip

Рассмотрим последовательность событий $A_n = \{w : \xi(w) <
x_n\}$

\includegraphics[width=60mm]{"1.png"}

Для неё верно:

\[
    \forall n \;\;\; A_n \subset  A_{n + 1} \;\;\; (x_n < x_{n + 1})  
\]

\[
    \bigcup\limits_{n = 1}^{\infty} A_n = A = \{w : \xi(w) < x_n\}  
\]

То есть последовательность $\{A_n\}$ удовлетворяет свойству
непрерывности вероятностной меры $P\left(\bigcup\limits_{n = 1}^{\infty} A_n\right) =
\lim\limits_{n \to \infty} P(A)$.

Тогда можем записать

\[
    \lim\limits_{n \to \infty} F(x_n) = \lim\limits_{n \to \infty}
    P\{\xi \in (-\infty, x_n)\} = \lim\limits_{n \to \infty}
    P\{\xi \in (-\infty, x)\} = F(x).
\]

Таким образом,

\[
    \lim\limits_{n \to \infty} F(x_n) = F(x).
\]
\bigskip

\subsection{Билет 4}      

\textbf{Доказательство неубывания функции распределения:}
\smallskip

По определению требуется показать, что:

\[
    \forall x_1 < x_2 \;\;\; F(x_1) \leq F(x_2)  
\]

Пусть 

\[
    (-\infty, x_2) = (-\infty, x_1) \cup [x_1, x_2]  
\]

Тогда

\[
    F(x_2) = P\{\xi \in (-\infty, x_2)\} =
    P\{\xi \in (-\infty, x_1)
    \cup \xi \in [x_1, x_2)\} = P\{\xi \in (-\infty, x_1)\} +
    P\{\xi \in [x_1, x_2]\}  
\]

Учитывая, что $P\{\xi \in (-\infty, x_1)\} \geq 0$ и 
$P\{\xi \in [x_1, x_2]\} \geq 0$, получим

\[
    P\{\xi \in (-\infty, x_1)\} + P\{\xi \in [x_1, x_2]\} \geq
    P\{\xi \in (-\infty, x_1)\} = F(x_1)
\]

То есть

\[
    \forall x_1 < x_2 \;\;\; F(x_1) \leq F(x_2)  
\]
\bigskip

\subsection{Билет 5}

\textbf{Определение функции плотности
распределения:}
\smallskip

Функцией плотности распределения вероятностей
случайной величины $\xi$ называется функция
$f_{\xi}(x)$ такая, что:

\begin{enumerate}
    \item{$\forall x \in R \;\;\;\; f_{\xi}
    (x) \geq 0$;}
    \item{$F_{\xi}^{'}(x) = f_{\xi}(x)$ почти всюду;}
    \item{$\int\limits_{-\infty}^{\infty} f_{\xi}(x)
    dx = 1$;}
    \item{$P\{a \leq \xi < b\} = \int\limits_b^{a}
    f_{\xi}(x) dx$.}
\end{enumerate}
\bigskip        

\subsection{Билет 6}

\textbf{Определение случайных независимых
велечин:}
\smallskip

Случайные велечины $\xi$ и $\mu$ называются
независимыми, если:

\[ 
    \forall x, y \in R \;\;\;\; P\{w : \xi(w)
    < x; \mu (w) < y\} = P\{w: \xi (w) < x\} \cdot
    P\{w : \mu (w) < y\}
\]
\bigskip

\subsection{Билет 7}

Пусть $(\Omega, \mathcal{F}, P)$ "--- вероятностное пространство
и пусть $\xi$ "--- случайная величина на нём.

\[
\xi = \xi(w) \;\;\; P = P(w)
\]
\bigskip

\textbf{Определение математического ожидания:}
\smallskip

Математическим ожиданием случайной величины $\xi$ называется

\[
    M\xi = \int\limits_{\Omega} \xi(w) d P(w)  
\]

Пусть для $\xi$ построена функция распределения $F_{\xi}(x) =
P\{\xi < x\}$.

Тогда

\[
    M\xi = \int\limits^{\infty}_{-\infty}  x d F_{\xi} (x)  
\]

Для дискретной случайной величины математическое ожидание 
находится по формуле:

\[
    M\xi = \sum^{\infty}_{i = 1} x_i p_i  
\]

Для абсолютно непрерывной величины:

\[
    M\xi = \int\limits^{\infty}_{-\infty} x f_{\xi} (x) dx  
\]

\textbf{Свойствай математического ожидания:}
\smallskip    

\begin{enumerate}
    \item{$Mc = c$, $c$ "--- const;}
    \item{$Mc\xi = cM\xi$;}
    \item{$M(\xi \pm \eta) = M\xi \pm M\eta$;}
    \item{Если $\xi$ и $\eta$ независимы, то 
    
    \[
        M\xi\eta = M\xi M\eta;   
    \]
    }
    \item{Если $\xi \geq 0 \;\;\; (P\{\xi \geq 0\} = 1)$, то
    $M\xi \geq 0$;}
    \item{Неравенство Коши"=Буняковского:
    
    \[
        M|\xi\eta| \leq M|\xi|M|\eta|;  
    \]
    }
    \item{Неравенство Чебышёва:
    
    Пусть $\xi$ "--- некоторая неотрицательная величина,
    а $g(x)$, неубывающая на множестве значений $\xi$, 
    непрерывная функиция. Тогда
    
    \[
        \forall \varepsilon > 0 \;\;\; P\{\xi \geq \varepsilon\} \leq
        \frac{Mg(\xi)}{\varepsilon};
    \]
    }
\end{enumerate}

\textbf{Определение дисперсии случайной величины:}
\smallskip

Пусть $\xi$ "--- случайная величина и $|M\xi| < +\infty$.
\bigskip

Дисперсией случайной величины $\xi$ называется число 

\[
D\xi = M(\xi - M\xi)^2
\]

\textbf{Свойства дисперсии:}
\smallskip

\begin{enumerate}
    \item{$D\xi \geq 0$;}
    \item{$Dc = 0$;}
    \item{$Dc\xi = c^2 D\xi$;}
    \item{Если случайные величины $\xi$ и $\eta$ независимы, то
    
    \[
        D(\xi \pm \eta) = D\xi \pm D\eta;
    \]
    }
    \item{$D\xi = M\xi^{2} - (M\xi)^2$;}
    \item{Для произвольных случайных величин $\xi$ и $\eta$ с
    $M|\xi| < + \infty$ и $M|\eta| < + \infty$ верно
    
    \[
        D(\xi \pm \eta) = D\xi \pm D\eta \pm {\text{cov}(\xi, \eta)},
    \]

    где ${\text{cov}(\xi, \eta)} = M(\xi - M\xi)(\eta - M\eta)$ "--- ковариация.
    }
\end{enumerate}

\subsection{Билет 8}        

\textbf{Определение дискретной случайной величины:}
\smallskip

Дискретной случайной величиной называется случайная величина,
множество значений которой конечно или счётно, то есть

\[
    \xi \in \{x_1, x_2, \dots\}  
\]

\subsection{Билет 9}

\textbf{Распределение Бернулли $(\xi \sim Bern(p))$.}
\bigskip

\textbf{Используемые обозначения:}
\smallskip

\begin{enumerate}
    \item{$A$ "--- <<успех>>;}
    \item{$\overline{A}$ "--- <<неуспех>>;}
    \item{$\xi = 1$, если A, в противном случае $\xi = 0$;}
    \item{p "--- вероятность наступления $A$;}
    \item{q "--- вероятность наступления $\overline{A}$.}
\end{enumerate}
\bigskip

\textbf{Закон распределения:}
\smallskip

\[
    P\{\xi = k\} = p^k q^{n - k},  
\]
где $k = 0,\ 1$.
\bigskip

\textbf{Математическое ожидание случайной величины $\xi$:}
\smallskip

\[
    M\xi = 0\cdot q + 1\cdot p = p  
\]
\bigskip

\textbf{Математическое ожидание случайной величины $\xi^2$:}
\smallskip

\[
    M\xi^2 = 0\cdot q + 1\cdot p = p  
\]
\bigskip

\textbf{Дисперсия:}
\smallskip 

\[
    D\xi = M\xi^2 - (M\xi)^2 = p - p^2 = p (1 - p) = pq.  
\]

\subsection{Билет 10}

\textbf{Биномиальное распределение $(\xi \sim Bin(n; p))$.}
\bigskip

\textbf{Используемые обозначения:}
    \smallskip

    \begin{enumerate}
        \item{$A$ "--- <<успех>>;}
        \item{$\overline{A}$ "--- <<неуспех>>;}
        \item{$\xi = 1$, если A, в противном случае $\xi = 0$;}
        \item{p "--- вероятность наступления $A$;}
        \item{q "--- вероятность наступления $\overline{A}$.}
    \end{enumerate}
    \bigskip

\textbf{Закон распределения:}
\smallskip

\[
    P\{\xi = k\} = C^k_n p^k q^{n - k},
\]

где $k = 0, 1, \dots, n$.
\bigskip

\textbf{Представление случайной величины $\xi$:}
\smallskip

Случайная велична $\xi$ может быть представлена 
в виде суммы случайных величин $\xi_i \sim Bern(p)$, каждая из
которых выражает результа $i$"=го испытания.

\[
    \xi = \xi_1 + \xi_2 + \dots + \xi_n.
\]
\bigskip

\textbf{Математическое ожидание случайной величины $\xi$:}
\smallskip

\[
    M\xi = M(\xi_1 + \xi_2 + \dots + \xi_n) =
    M(\xi_1) + M(\xi_2) + \dots + M(\xi_n) =
    p + p + \dots + p = np.
\]
\bigskip

\textbf{Дисперсия случайной величины $\xi$:}
\smallskip

\[
    D\xi = D(\xi_1 + \xi_2 + \dots + \xi_n) =
    D(\xi_1) + D(\xi_2) + \dots + D(\xi_n) =
    pq + pq + \dots + pq = npq.
\]
\bigskip

\textbf{Схема Бернулли:}
\smallskip

Эксперимент, удовлетворяющий следующим условиям,
называетcя схемой испытаний Бернулли:

\begin{enumerate}
    \item{Случайная величина $\xi$ "--- количество успехов
    в $n$ независимых одинаковых испытаниях;}
    \item{                     
    Одинаковые: $P\{\xi_i = 1\} = p(i) \simeq p$
    (практически не зависящие от номера испытания);}
    \item{Случайные величины $\xi_i \sim Bern(p)$
    независимы:
    
    \[
        P\{\xi_i = 1\cap \xi_j = 1\} =
        P\{\xi_i = 1\} P\{\xi_j = 1\} \;\;\; 
        \forall i \neq j.  
    \]
    }
\end{enumerate}

\subsection{Билет 11}

Докажем, что $P\{\xi = k\} = C^k_n p ^k q^{n - k}$.
\bigskip

Пусть $\Omega = \{\overline{w} = (\xi_1, \xi_2, \dots, 
\xi_n)\}$, где $\xi_i \sim Bern(p); \;\;\; \xi_i \in
{0, 1}$, то есть $\overline{w}$ "--- последовательность 
$n$ нулей или единиц "--- результатов испытаний Бернулли.
\bigskip

Событие $\{\xi = k\} = \{\overline{w} = (\xi_1, \xi_2, \dots, 
\xi_n) : \xi_1 + \xi_2 + \dots + \xi_n = k\}$ состоит
из $\overline{w}$ "--- векторов, состоящих из k "--- единиц и (n - k) "---
нулей.
\bigskip

Пусть $\overline{w^{*}} = (1, 1, \dots, 1, 0, 0, \dots, 0)$ "---
фиксированный исход, в котором количество единиц = k, а нулей = n - k.
\smallskip

Найдём верояность этого исхода.
\smallskip

\[
p^{*} = P\{\xi_1 = 1; \dots ; \xi_k = 1; \xi_{k + 1} =
0; \dots ; \xi_n = 0\} = 
\]
\[
= P\{\xi_1 = 1\}\cdot\dots\cdot
P\{\xi_k = 1\}\cdot P\{\xi_{k + 1} = 0\}\cdot\dots\cdot
P\{\xi_n = 0\} = p^k q^{n - k}
\]
\smallskip

Здесь воспользовались незвисимостью испытаний и тем, что 
испытания предполагаются одинаковыми, то есть $p(n) = p$ вероятность
<<успеха>> не зависит от номера шага.
\bigskip

Заметим, что в остальных $\overline{w} \in A$ также k единиц и (n - k)
нулей, а значит у каждого $\overline{w} \in A$ вероятность 
$p^k q^{n - k}$.
\bigskip

Определим количество исходов в A. Оно будет равно количеству способов
выбрать k"=й мест из n для того, чтобы поставить на них 1. Остальные
позиции среди n мест заполняются нулями.
\smallskip

Отсюда делаем вывод, что количество равно $C^k_n$.
\smallskip

Следовательно,

\[
P\{\xi = k\} = P\{\overline{w}: \xi_1 + \xi_2 + \dots +
\xi_n = k\} = \sum_{\overline{w} \in A} P\{\overline{w}\} =
C^k_n p^k q^{n - k}.
\]

\subsection{Билет 12}

\textbf{Определение абсолютно непрерывной случайной величины:}
\smallskip

Случайная величина $\xi$ называется абсолютно непрерывной,
если существует такая неотрицательная функция $f(x)$, что

\[
    \forall x \in R \;\;\; F(x) = \int\limits^x_{-\infty} f(t) dt.  
\]
\bigskip

\subsection{Билет 13}

\textbf{Нормальное распределение $\xi \sim N(a, \sigma^2)$.}
\bigskip

\textbf{Функция плотности случайной величины $\xi \sim N(a,
\sigma^2)$:}
\smallskip

\[
    f(x) = \frac{1}{\sqrt{2 \pi \sigma^2}}
    {\text{exp}\left(-\frac{(x - a)^2}{2 \sigma^2}\right)}
\]

\begin{center}
    \includegraphics[width=60mm]{"2.png"}
\end{center}
\bigskip

\textbf{Функция плотности случайной величины $\xi_0 \sim N(0, 1)$:}
\smallskip   

\[
    f(x) = \frac{1}{\sqrt{2 \pi}}
    {\text{exp}\left(-\frac{x^2}{2}\right)}
\]
\bigskip

\textbf{Математическим ожиданием стандарной нормальной
случайной величины $\xi_0 \sim N(0, 1)$:}
\smallskip     

\[
    M\xi_0 = \int\limits^{+\infty}_{-\infty} x e^{-\frac{x^2}{2}} dx =
    \int\limits^0_{-\infty} x e^{-\frac{x^2}{2}} dx +
    \int\limits^{+\infty}_{0} x e^{-\frac{x^2}{2}} dx = 0
\]
\bigskip

\textbf{Математическим ожиданием случайной величины $\xi_0^2$:}
\smallskip     

\[
    M\xi_0^2 = \int\limits^{+\infty}_{-\infty} x e^{-\frac{x^2}{2}} dx = 1
\]
\bigskip

\textbf{Математическим ожиданием случайной величины $\xi$:}
\smallskip     

\[
    M\xi = M(\sigma \xi_0 + a) = a
\]
\bigskip    

\textbf{Дисперсия случайной величины $\xi_0 \sim N(0, 1)$:}  
\smallskip

\[
    D\xi_0 = M\xi^2_0 - (M\xi_0)^2 = 1 - 0 = 1.  
\]

\textbf{Дисперсия случайной величины $\xi \sim N(a,
\sigma^2)$:}  
\smallskip

\[
    D\xi = D(\sigma\xi_0 + a) = \sigma^2.  
\]  
\bigskip

\subsection{Билет 14}

\textbf{Равномерное распределение $\xi \sim R[a, b]$.}
\bigskip

\textbf{Функция плотности случайной величины $\xi$:}
\smallskip

\begin{equation*}
    f(x)=
    \begin{cases}
        0, \;\;\;\; x \notin [a, b];\\
        1/(b - a), \;\;\;\; x \in [a, b]\\
    \end{cases}
\end{equation*}

\begin{center}
    \includegraphics[width=60mm]{"3.png"}
\end{center}
\bigskip

\textbf{Математичяеское ожидание случайной величины $\xi$:}
\smallskip

\[
    M\xi = \int\limits^a_b x \frac{1}{b - a}dx = \frac{b^2 - a^2}{2(b - a)}=
    \frac{b + a}{2}.  
\]
\bigskip

\textbf{Математическое ожидание случаной величины $\xi^2$:}
\smallskip

\[
    M\xi^2 = \int\limits^a_b x^2 \frac{1}{b - a}dx = 
    \frac{b^3 - a^3}{3(b - a)} = \frac{b^2 + ab + a^2}{3}.  
\]
\bigskip

\textbf{Дисперсия случайной величины $\xi$:}
\smallskip

\[
    D\xi = M\xi^2 - (M\xi)^2 = \frac{(b + a)^2}{4} -
    \frac{b^2 + ab + a^2}{3} = \frac{(b - a)^2}{12}.  
\]
\bigskip

\subsection{Билет 15}

\textbf{Определение случайных независимых
велечин:}
\smallskip

Случайные велечины $\xi$ и $\mu$ называются
независимыми, если:

\[ 
    \forall x, y \in R \;\;\;\; P\{w : \xi(w)
    < x; \mu (w) < y\} = P\{w: \xi (w) < x\} \cdot
    P\{w : \mu (w) < y\}
\]
\bigskip

\textbf{Критерий независимости дискретной случаной величины:}
\smallskip

Дискретные случайные величины $\xi$ и $\eta$ независимы, тогда
и только тогда, когда:

\[
    \forall i \neq j \;\;\; p_{ij} = p_i p_j  
\]
\bigskip

\textbf{Критерий независимости абсолютно непрерывной случаной величины:}
\smallskip

Абсолютно непрерывные дискретные случайные величины $\xi$ и
$\eta$ независимы, тогда и только тогда, когда:

\[
    f_{\xi \eta}(x, y) = f_{\xi} (x) f_{\eta} (y). 
\]    

        
\end{document}