\documentclass[14pt]{extarticle}

\usepackage[T2A]{fontenc}
\usepackage[cp1251]{inputenc}
\usepackage[russian]{babel}

\usepackage{graphicx}

\usepackage{amsmath, amsthm, amsfonts, amssymb}
\usepackage[unicode, hidelinks]{hyperref}

\usepackage[
    a4paper, mag=1000,
    left=2.5cm, right=1.5cm, top=2cm, bottom=2cm, bindingoffset=0cm,
    headheight=0cm, footskip=1cm, headsep=0cm
]{geometry}

\usepackage{tempora}

\usepackage{chngcntr} % equation counter manipulations

% reset equation counter at each section
\counterwithin*{equation}{section}
\counterwithin*{equation}{subsection}

\graphicspath{{images/}}

\usepackage{cases}

\begin{document}
    
\tableofcontents

\newpage

\section{Глава 1}

\subsection{Случайные события, классификация событий, операции над ними.}

\textbf{Определение случайного события:}
    \smallskip

    Пусть $\Omega$ "--- множество элементарных
    исходов эксперимента. Случайным событием
    называется любое подмножество множества
    $\Omega$.
    \bigskip

\textbf{Определение достоверного события:}
    \smallskip

    Достоверным событием называется событие
    $\Omega$, которому благоприятствует
    каждый исход эксперимента.
    \bigskip

\textbf{Определение невозможного события:}
    \smallskip
    
    Невозможным событием называется пустое
    множество, которому не благоприятствует ни
    один исход эксперимента.
    \bigskip

\textbf{Определение суммы событий:}
    \smallskip

    Суммой событий $A$ и $B$ называется событие $C = A \cup B$,
    которому благоприятствуют исходы, принадлежащие хоть одному
    из событий $A$ или $B$.
    \bigskip

\textbf{Определение произведения событий:}
    \smallskip
    
    Произведением событий $A$ и $B$ называется событие 
    $C = A \cap B$, которому благоприятствуют исходы 
    и события $A$, и события $B$.
    \bigskip

\textbf{Определение несовместных событий:}
    \smallskip

    Случайные события $A$ и $B$ называются несовместными, если 
    $A \cap B = \varnothing$.
    \bigskip

\textbf{Определение противоположного события:}
    \smallskip
    
    Событием, противоположным событию 
    $A$ называется событие $\overline{A}$, которое
    состоит из исходов, не благоприятствующих $A$.
    \bigskip

\subsection{Определения: кольцо, алгебра, $\sigma$"=алгебра, минимальная 
$\sigma$"=алгебра над классом $K$. Борелевская $\sigma$"=алгебра.}     

\textbf{Определение кольца:}
    \smallskip

    Кольцом $\mathbf{R}$ называется непустой класс множества
    замкнутый относительно операций сложения и взятия разности.
    \bigskip

\textbf{Определение алгебры:}
    \smallskip

    Алгеброй $\mathcal{A}$ называется непустой класс множества
    замкнутый относительно сложения и отрицания.
    \bigskip
    
\textbf{Определение $\sigma$"=алгебры:}
    \smallskip

    $\sigma$"=алгебра $\mathcal{F}$~--- это непустой класс множества
    замкнутый относительно счётного количества сумм и отрицаний:

    \begin{enumerate}
        \item{Если $A \in \mathcal{F}$, то $\overline{A} \in \mathcal{F}$;}
        \item{$\Omega \in \mathcal{F}$;}
        \item{Если $\{A_i\}^{\infty}_{i = 1} \in \mathcal{F}$,
        то $\bigcup^{\infty}_{i = 1} A_i \in \mathcal{F}$.}
    \end{enumerate}
    \bigskip

\textbf{Определение $\sigma$"=алгебры событий:}
    \smallskip

    Сигма алгеброй событий называется множество
    $\mathcal{F}$ подмножеств $A \subset \Omega$,
    удовлетворяющее условиям:

    \begin{enumerate}
        \item{если $A \in \mathcal{F}$, то
        $\overline{A} \in \mathcal{F}$;}
        \item{$\Omega \in \mathcal{F}$;}
        \item{если $\{A\}_{i = 1}^{\infty} \in
        \mathcal{F}$, то
        $\bigcup\limits_{i = 1}^{\infty} A_i \in 
        \mathcal{F}$.}
    \end{enumerate}
    \bigskip    

\textbf{Определение минмальной $\sigma$"=алгебры над классом $K$:}
    \smallskip
    
    Пусть $K$~--- некоторый класс подмножеств из $\Omega$. $\sigma$"=алгебра
    $\sigma(K)$ называется наименьшей $\sigma$"=алгеброй, содержащей
    класс $K$, если $K \in \sigma(K)$; любая $\sigma$"=алгебра $\mathcal{F}$,
    которая содержит $K \;\; (K \subset \mathcal{F})$, содержит и 
    $\sigma(K) \subset \mathcal{F}$.
    \bigskip

\textbf{Определение Борелевской $\sigma$"=алгебры:}
    \smallskip
    
    Борелевской $\sigma$"=алгеброй $\beta$ называется минимальная $\sigma$"=алгебра
    над классом полуинтервалов $K = \{[a, b]\}$ из $R$, то есть:

    \[
        \Omega = (-\infty, \infty) = R \;\;\; K = \{[a, b), [a, +\infty),
        (a, +\infty), (-\infty, b), (-\infty, b], (a, b]\}.  
    \]
    \bigskip

\subsection{Теорема Каратеодори.}

    Пусть $Q(A)$~--- счётно аддитивная вероятностная мера на алгебре $\mathcal{A}$.
    Тогда существует единственная счётно аддитивная вероятностная мера $P(A)$, заданная
    на минимальной $\sigma$"=алгебре $\mathcal{F}$ и являющаяся её продолжением,
    то есть $\forall A \in \mathcal{A} \;\;\; P(A) = Q(A).$
    \bigskip

\subsection{Определения: мера, конечно"=аддитивная, счётно"=аддитивная мера}

Пусть $\Omega$~--- множество элементарных исходов эксперимента. Некоторое 
его подмножество $A \subset \Omega$ называется случайным событием.
\bigskip

\textbf{Определение конечно аддитивной
вероятностной меры}
    \smallskip

    Конечно аддитивной вероятностной мерой $Q(A)$
    называется функция множества $Q: \mathcal{A}
    \rightarrow [0; 1]$, такая, что:

    \begin{enumerate}
        \item{$\forall A \in \mathcal{A} \;\;\;
        Q(A) \geq 0$;}
        \item{$Q(\Omega) = 1$;}
        \item{$\forall A, B \in \mathcal{A} : 
        \;\;\; A \cap B = \varnothing \;\;\;
        Q(A \cup B) = Q(A) + Q(B) \;\;\;
        Q\left(\bigcup\limits_{i = 1}^{\infty} A_i\right) = \sum_{i = 1}^{\infty}
        Q(A_i)$.}
    \end{enumerate}
    \bigskip  

\textbf{Определение счётно аддитивной 
вероятностной меры:}
    \smallskip

    Счётно аддитивно вероятностной мерой $P(A)$
    называется функция множества $P: \mathcal{F}
    \rightarrow [0; 1]$, такая, что:

    \begin{enumerate}
        \item{$\forall A \in \mathcal{F} \;\;\;
        P(A) \geq 0$;}
        \item{$P(\Omega) = 1$;}
        \item{$\forall \{A_i\}_{i = 1}^{\infty}
        \in \mathcal{F} : \;\;\; \forall
        i \neq j \;\;\; A_i \bigcap A_j = 
        \varnothing \;\;\; P\left(\bigcup\limits_{i = 1}^{
        \infty} A_i\right) = \sum_{i = 1}^{\infty}
        P(A_i)$.}
    \end{enumerate}
    \bigskip  

\subsection{Построение меры Лебега. Верхняя мера Лебега, нижняя мера Лебега, мера
Лебега. Измеримое по Лебегу множество.}    

Пусть $P = <a, b> \times <c, d>$. $P \subset R^2$~--- прямоугольник.
\bigskip

Мерой прямоугольника назовём $m(P)$, где

$m(P) = (b - a)(d - c)$

\begin{center}
    \includegraphics[width=100mm]{"5.1.png"}
\end{center}
\bigskip

Множество $A$ назовём эелментарным, если оно представимо в виде
суммы прямоугольников хотя бы 1 способом:

\[
    A = \bigcup P_k,  
\]

где $\{P_k\}$~--- покрытие.
\bigskip

Мерой элементарного сножества $A$ называется 

\[
    m^{'}(A) = \sum m(P_k),  
\]

где $\{P_k\}$~--- разбиение $A$, то есть $\forall j \neq k \;\;\; P_k \cap P_j =
\varnothing$.
\bigskip

Рассмотрим множество $E = [0; 1] \times [0; 1]$

\begin{center}
    \includegraphics[width=100mm]{"5.2.png"}
\end{center}
\bigskip

\textbf{Определение верхней меры Лебега:}
    \smallskip    

    Пусть $A$~--- некоторое множество. Рассмотрим $\{P_k\}$, такое, что:

    \[
        A \subset P_k
    \]

    \begin{center}
        \includegraphics[width=100mm]{"5.3.png"}
    \end{center}

    Верхней мерой Лебега называется 

    \[
        \mu^{*}(A) = \inf_{\{P_k\}} \sum m(P_k)
    \]
    \bigskip

\textbf{Определение нижней меры Лебега:}
    \smallskip
    
    Рассмотрим множество $E \backslash A. (m(E) = 1)$

    \begin{center}
        \includegraphics[width=100mm]{"5.4.png"}
    \end{center}

    Нижней мерой Лебега называется:

    \[
        \mu_{*}(A) = 1 - \mu^{*}(E \backslash A)  
    \]
    \bigskip
    
\textbf{Определение меры Лебега и измеримого по Лебегу множества:}
    \smallskip
    
    Говорят, что множество $A$ измеримо по Лебегу, если $\mu^{*}(A) =
    \mu_{*}(A) = \mu (A)$.

    Величина $\mu(A)$~--- называется мерой Лебега множества $A$.

\subsection{Вероятностная мера, её свойства, непрерывность вероятностной меры.}

\textbf{Определение вероятностной меры:}
    \smallskip

    Вероятностной мерой называется функция $P : \mathcal{F} \to [0, 1]$,
    удовлетворяющая условиям:

    \begin{enumerate}
        \item{$P(\Omega) = 1$;}
        \item{$\forall A \in \mathcal{F} \;\;\; P(A) \geq 0$;}
        \item{$\forall \{A_i\}^{\infty}_{i = 1} \in \mathcal{F}$,
        такой, что $\forall i \neq j \;\;\; A_i \cap A_j = \varnothing$,
        $P\left(\bigcup\limits^{\infty}_{i = 1}A_i\right) = \sum\limits^{\infty}_{i = 1} P(A_i)$.}
    \end{enumerate}
    \bigskip

\textbf{Свойства вероятностной меры:}    
    \smallskip

    \begin{enumerate}
        \item{
        \begin{center}
            \includegraphics[width=100mm]{"6.1.png"}
        \end{center}
        }
        \item{
        \begin{center}
            \includegraphics[width=100mm]{"6.2.png"}
        \end{center}
        }
        \item{
        \begin{center}
            \includegraphics[width=100mm]{"6.3.png"}
        \end{center}
        }
        \item{
        \begin{center}
            \includegraphics[width=100mm]{"6.4.png"}

            \includegraphics[width=120mm]{"6.5.png"}
        \end{center}
        }
    \end{enumerate}

\subsection{Классическое вероятностное пространство. Классическое определение
вероятности.}

Вероятностной моделью стохастического эксперимента называется тройка
$(\Omega, \mathcal{F}, P)$, где $\Omega$~--- множество элементарных
исходов экмперимента, $\mathcal{F}$~--- алгебра событий, 
$P$~--- вероятностная мера.

$(\Omega, \mathcal{F}, P)$~--- вероятностное пространство.
\bigskip

\textbf{Определение классического вероятностного пространства:}
    \smallskip

    Классическим вероятностым пространством, называется вероятностное
    пространство $(\Omega, \mathcal{F}, P)$, в конечном множестве 
    элементарных исходов которого все элементарные исходы равновозможны.
    \bigskip

    Построим вероятностную меру:
    \bigskip

    Пусть $\Omega = \{w1, \dots, w_n\}$.

    Рассмотрим $\{ A_i \}^{n}_{i = 1}$, где $A_i = \{w_i\}$. Тогда

    \[
        A_i \cap A_j = \varnothing \;\;\; \bigcup^{n}_{i = 1} A_i = \Omega.  
    \]

    \[
        1 = P(\Omega) = P(\bigcup^{n}_{i = 1} A_i) = \sum^{n}_{i = 1} P(A_i) =
        | P(A_i) = P(w_i) = p | = \sum^{n}_{i = 1} p \Rightarrow
    \]
    \[
        \Rightarrow p = \frac{1}{n} \Rightarrow \forall w_i \;\;\; P(w_i) =
        \frac{1}{n}  
    \]

    Пусть $A = \{w_{i_1}, \dots, w_{i_k}\}$ $0 \leq k \leq n$. Тогда
    вероятностная мера в классическом вероятностном пространстве имеет
    вид 

    \[
        P(A) = P \left( \bigsqcup^{k}_{j = 1} w_{ij} \right) =
        \sum^{k}_{j = 1} P(w_{ij}) = \sum^{k}_{j = 1} \frac{1}{n} = \frac{k}{n},  
    \]
    \bigskip

    \textbf{$P(A) = \frac{k}{n}$~--- называется классической вероятностью},

    где $k$~--- количество благоприятных $A$ элементарных исходов,
    $n$~--- количество элементарных исходов эксперимента.

\subsection{Дискретное вероятностное пространство.}

\textbf{Определение дискретного вероятностного пространства:}
    \smallskip

    Дискретным вероятностным пространством называется вероятностное
    пространство $(\Omega, \mathcal{F}, P)$, такое, что 
    $\Omega$~--- конечное или счётное множество неравновозможных исходов.
    \bigskip

    Вероятностную меру зададим числами $p_i = P(w_i) > 0$, такими, что
    $\sum^{\infty}_{i = 1} p_i = 1$. Тогда $\forall A \in \mathcal{F}$
    веротность вычисляется как $P(A) = P(\bigcup\limits_{w_i \in A} w_i) =
    \sum\limits_{w_i \in A} P (w_i)$.

\subsection{Условная вероятность. Теорема умножения вероятностей.}    

\textbf{Определение условной вероятности:}
    \smallskip

    Пусть $(\Omega, \mathcal{F}, P)$~--- вероятностное пространство и
    $A, B \in \mathcal{F}; \;\;\; P(B) > 0$. 

    Условной вероятностью события $A$ при условии, что наступило
    событие $B$ называется число:

    \[
        P(A|B) = \frac{P(A \cap B)}{P(B)}.  
    \]
    \bigskip

\textbf{Теорема умножения вероятностей:}
    \smallskip
    
    Пусть $A$ и $B$~--- случайные события и $P(B) > 0$. Тогда 

    \[
        P(A \cap B) = P(B) \cdot P(A|B)  
    \]
    \bigskip

    Пусть $A_1, A_2, A_3$~--- случайные события и $P(A_1) > 0$ и $P(A_1 \cap A_2) > 0$.
    Тогда

    \[
        P(A_1 \cap A_2 \cap A_3) = P(A_1) \cdot P(A_2|A_1) \cdot P(A_3|A_1 \cap A_2)  
    \]

\subsection{Формулы полной вероятности и Байеса.}

\textbf{Теорема (формула полной вероятности):}
    \smallskip

    Пусть $(\Omega, \mathcal{F}, P)$~--- вероятностное пространство и
    $\{A_i\}^{\infty}_{i = 1} \in \mathcal{F}$~--- полная группа
    попарно несовместных событий; $P(A_i) \geq 0$. Пусть $A \in \mathcal{F}$~---
    неполное событие $P(A|A_i) \geq 0$. Тогда $P(A) = \sum^{\infty}_{i = 1} P(A_i)
    \cdot P(A|A_i)$.
    \bigskip

    Доказательство:
    \bigskip

    $$P(A) = P(A \cap \Omega) = P(A \cap (\sqcup^{\infty}_{i = 1} A_i)) =
    P(\sqcup^{\infty}_{i = 1}(A \cap A_i)) \Rightarrow$$

    $$\Rightarrow \sum^{\infty}_{i = 1} P (A \cap A_i) = \sum^{\infty}_{i = 1}
    P(A_i)P(A|A_i)$$.
    \bigskip

\textbf{Теорема (формула Байеса):}
    \smallskip
    
    Пусть $\{A_i\}^{\infty}_{i = 1} \in \mathcal{F}$~--- полная группа
    попарно несовместных событий и пусть для некоторого $P(A) > 0$. Тогда

    \[
        \forall i = \overline{1, \infty} \;\;\; P(A_i|A) = 
        \frac{P(A_i)P(A|A_i)}{P(A)}  
    \]
    \bigskip

    Доказательство:
    \bigskip

    \[
        P(A_i|A) = \frac{P(A \cap A_i)}{P(A)} = \frac{P(A_i) \cdot P(A|A_i)}{P(A)}  
    \]

\subsection{Независимость событий. Независимость в совокупности.}

\textbf{Определение независимости событий:}
    \smallskip

    Случайные события $A$ и $B$ называются независимыми, если:

    \[
        P(A \cap B) = P(A) \cdot P(B)   
    \]
    \bigskip

\textbf{Определение независимости в совокупности:}
    \smallskip

    $\{A_i\}^{n}_{i = 1}$~--- называются независимыми в совокупности, если

    \[
        \forall 2 \leq k \leq n \;\;\; P (\bigcap^{k}_{j = 1}A_{ij}) =
        \prod^{k}_{j = 1} P(A_{ij}) 
    \]

\subsection{Теорема о независимости противоположных событий. Критерий
независимости случайных событий.}

\textbf{Теорема о независимости противоположных событий:}
    \smallskip

    Пусть $A$ и $B$~--- независимы. Тогда события $A$ и $\overline{B}$,
    $\overline{A}$ и $B$. $\overline{A}$ и $\overline{B}$~--- попарно
    независимы.
    \bigskip

    Доказательство:
    \bigskip

    Рассмотрим $A$ и $\overline{B}$. Тогда $P(A \cap 
    \overline{B})$. Можем заметить, что $P(A \cap \overline{B}) = 
    P(A \backslash (A \cap B)) = P(A) - P(A \cap B) = P(A) - P(A) P(B) =
    P(A)(1 - P(B)) = P(A) P(\overline{B})$.

    Остальные случаи аналогичны.
    \bigskip

\textbf{Критерий независимости случайных событий:}
    \smallskip

    Пусть $A$ и $B$ такие, что $P(B) > 0$. Тогда Случайные
    события $A$ и $B$ независимы, тогда и только тогда, когда:

    \[
        P(A|B) = P(A)  
    \]
    \bigskip

    Доказательство:
    \bigskip

    Необходимость:
    \smallskip

    Пусть $A$ и $B$ независимы:

    \[
        P(A|B) = \frac{P(A \cap B)}{P(B)} = \frac{P(A) \cdot P(B)}{P(B)} = P(A)  
    \]
    \bigskip

    Достаточность:
    \smallskip

    Пусть выполняется: $P(A|B) = P(A)$. Тогда из определения условной вероятности
    следует, что $P(A \cap B) = P(A|B) \cdot P(A) = P(A) \cdot P(B)$,
    то есть выполняется определение.

\section{Глава 2}

\subsection{Определение измеримой функции, абстрактной и действительной.
Критерий измеримости действительных функций.}

\textbf{Определение измеримой функции:}
    \smallskip

    Пусть $X$ и $Y$ "--- некоторые множества
    и пусть $S_x$ и $S_y$ "--- классы подмножества.
    $f: X \rightarrow Y$ "--- некоторая функция.
    \smallskip

    Функция $f: X \rightarrow Y$ называется 
    $(S_x, S_y)$ "--- измеримой, если:

    \[
        \forall B \in S_y \;\;\;\; \exists f^{-1}
        (B) \in S_x
    \]
    \bigskip

\textbf{Определение измеримой действительной
функции:}
    \smallskip
    
    Действительная функция $f(x)$ с областью 
    определения $X \subset R$ называется 
    $\mu$"=измеримой или $S_{\mu}$"=измеримой,
    если для любого борелевского множества $b \in
    \beta(R) \;\;\;\; f^{-1}(b) \in S_{\mu}$.
    \bigskip

\textbf{Определение (критерий измеримости 
действительных функций):}
    \smallskip

    Действительная функция $f(x)$ измерима
    $\Leftrightarrow$

    \begin{center}

        $
            \forall C \in R \;\;\;\; f^{-1}(b) = 
            f^{-1}(-\infty, C)
        $ "--- измерима

        или 

        $\{x : f(x) < C\}$

    \end{center}
    \bigskip

\subsection{Случайная величина. Виды случайных величин (дискретная и 
абсолютно непрерывная).}

\textbf{Определение случайной величины:}
    \smallskip

    Пусть $(\Omega, \mathcal{F}, P)$ "---
    вероятностное пространство. Случайной 
    величиной называется вещественно значная
    функция $\xi$ такая, что

    \[
        \xi : \Omega \rightarrow R \;\;\;\;
        \forall x \in R \;\;\;\; \{w: \xi(w) < x\}
        \in \mathcal{F}
    \]
    \bigskip

\textbf{Определение дискретной случайной величины:}
    \smallskip
    
    Дискретной случайной величиной называется случайная величина $\xi$,
    множество значений которой конечно или счётно, то есть

    \[
        \xi \in \{x_1, x_2, \dots\}  
    \]
    \bigskip

\textbf{Определение абсолюьно непрерывной случайной величины:}
    \smallskip
    
    Абсолютно непрерывной случайной величиной называется случайная величина $\xi$,
    такая, что 

    \[
        F_{\xi}(x) = \int^{x}_{-\infty} f_{\xi}(t) dt  
    \]

\subsection{Функция распределения и её свойства.}

\textbf{Определение функции распределения:}
    \smallskip

    Функцией распределения вероятностей 
    случайной величины $\xi$ называется функция 
    $F_{\xi}(x) = P\{w : \xi (w) < x\}$
    \bigskip

\textbf{Свойства функции распределения:}
    \smallskip
    
    \begin{enumerate}
        \item{$0 \leq F_{\xi}(x) \leq 1 \;\;\;\;
        \forall x \in R$;}
        \item{$F_{\xi}(x)$ "--- неубывающая,
        непрерывная слева функция;}
        \item{$\lim\limits_{x \to +\infty} F_{\xi} (x) = 1$
        
        $\lim\limits_{x \to -\infty} F_{\xi} (x) = 0$;}
        \item{$P\{a \leq \xi < b\} = F_{\xi}(b) -
        F_{\xi}(a)$;}
        \item{$P\{\xi = x_0\} = F_{\xi}(x_0 + 0)
        - F_{\xi}(x_0)$.}
    \end{enumerate}
    \bigskip

\subsection{Теорема о существовании случайной величины, соответствующей
функции со свойствами функции распределения.}

\textbf{Теорема (о существовании случайной величины, заданной
функцией распределения):}
    \smallskip

    Пусть $F(x)$ принимает значение на $[0, 1]$, неубывающая и $F(-\infty) = 0$,
    $F(+\infty) = 1$. Тогда $\exists$ вероятностное пространство $(\Omega, \mathcal{F},
    P)$ и $\xi$ на нём, для которой $P\{\xi < x\} = F(x)$.

\subsection{Фукнция плотности распределения случайной величины и её свойства.}

\textbf{Определение функции плотности
распределения:}
    \smallskip

    Функцией плотности распределения вероятностей
    случайной величины $\xi$ называется функция
    $f_{\xi}(x)$ такая, что:

    \begin{enumerate}
        \item{$\forall x \in R \;\;\;\; f_{\xi}
        (x) \geq 0$;}
        \item{$F_{\xi}^{'}(x) = f_{\xi}(x)$;}
        \item{$\int\limits_{-\infty}^{\infty} f_{\xi}(x)
        dx = 1$;}
        \item{$P\{a \leq \xi \leq b\} = \int\limits_b^{a}
        f_{\xi}(x) dx$.}
    \end{enumerate}
    \bigskip

\subsection{Дискретная случайная величина. Основные типы дескретных распределений
(постановка задачи, закон распределения): распределение Бернулли, 
равномерное дискретное, биномиальное, пуассоновское, геометрическое распределения.}

\textbf{Распределение Бернулли $(\xi \sim Bern(p))$:}
    \smallskip

    Пусть в множестве $\Omega$ различают два события $A$ и $\overline{A}$.
    Случайное событие $A$~--- успех $P(A) = p$, $\overline{A}$~--- неудача
    $P(\overline{A}) = q$.
    \smallskip
    
    $A \cup \overline{A} = \Omega$ и $A \cap \overline{A} = \varnothing$, 
    следовательно, $p + q = 1$.
    \smallskip

    Пусть $\xi = 1$, если наступил успех и $\xi = 0$, если наступила неудача.
    Ряд распределения имеет вид:

    \begin{center}
        \begin{tabular}{|c|c|c|}
            \hline
            $\xi$ & 0 & 1 \\
            \hline
            $P$ & $p$ & $q$ \\
            \hline
        \end{tabular}
    \end{center}
    \smallskip

    Закон распределения: $P\{\xi = k\} = p^{k} q^{1 - k}, \;\;\; k \in \{0; 1\}$.
    \bigskip

\textbf{Биномиальное распределение $(\xi \sim Bin(n; p))$:}
    \smallskip
    
    Произведено $n$ независимых, одинаковых, испытаний Бернули.

    Вероятность успеха $p(n) \simeq p$~--- почти не зависит от номера
    испытания.

    $\Omega = \{\overline{w} = (\xi_1, \xi_2, \dots, \xi_n) :
    \xi_i \in \{0, 1\} \}$.
    \smallskip

    Введём случайную величину $\xi$~--- количество успехов в $n$ испытаниях
    Бернули. Тогда $\xi \in \{0, 1, \dots, n\}$.
    \smallskip

    Закон распределения: $P\{\xi = k\} = C^{k}_{n} p^{k} q^{n - k}$.
    \bigskip

\textbf{Пуассоновское распределение $(\xi \sim Pois(\lambda))$:}
    \smallskip
    
    Произведено большое количество испытаний Бернулли, то есть $Bin(n, p)$,
    но $n$~--- велико $\gg 1000$.

    Условие применения пуассоновского приближения: $p \leq 0,1; \;\;\; npq \leq 9$.
    \smallskip

    $\xi = 0, 1, \dots$
    \smallskip

    Закон распределения: $P_n(k) = \frac{e^{-\lambda} \lambda^{k}}{k!}$,
    где $\sum^{\infty}_{k = 0} \frac{e^{-\lambda} \lambda^k}{k!} = e^{-\lambda} 
    \cdot e^{\lambda} = 1$.
    \bigskip

\textbf{геометрическое распределение:}
    \smallskip

    Испытания производятся до тех пор пока не появится первый успех.
    \smallskip

    $\xi = 1, 2, \dots$~--- количество произведённых испытаний.
    \smallskip

    Закон распределения: $P\{\xi = k\} = p q^{k - 1} \;\;\; 0 < p < 1 \;\;\;
    q = 1 - p \;\;\; k = 1, 2, \dots$
    \bigskip

\textbf{Равномерное дискретное распределение $(\xi \sim R(N))$:}
    \smallskip
    
    Конечное множество равновозможных исходов (классическое вероятностное
    пространство).
    \smallskip

    $\xi$~--- номер наступившего исхода.
    \smallskip

    $P\{\xi = k\} = \frac{1}{N}; \;\;\; k = \overline{1, N}$,

    где $N$~--- общее количество исходов.

\subsection{Равномерное непрерывное распределение (построение функций 
распределения и плоности, графики).}

\textbf{Равномерное непрерывное распределение:}
    \smallskip

    \begin{center}
        \includegraphics[width=140mm]{"19.1.png"}
        \bigskip

        \includegraphics[width=120mm]{"19.2.png"}
        \bigskip
    \end{center}
\subsection{Показательное (экспоненциальное) распределение (построение функции
распределения, функции плотности, графики, свойство отсутствия последействия)}

\begin{center}
    \includegraphics[width=140mm]{"20.1.png"}
    \bigskip

    \includegraphics[width=140mm]{"20.2.png"}
    \bigskip
\end{center}

\textbf{Свойство отсутствия последействия:}
    \smallskip 

    \begin{center}
        \includegraphics[width=140mm]{"20.3.png"}
    \end{center}

\subsection{Нормальное распределение (функции распределения, функции плотности,
свойства).}

\textbf{$(\xi \sim N(a, \sigma^2))$:}
    \smallskip

    Функция распределения: $F_{\xi}(x) = \frac{1}{\sqrt{2 \pi} \sigma}
    \int\limits^{x}_{-\infty} e^{-\frac{(t - a)^2}{2 \sigma^2}}dt$
    \smallskip

    Функция плотности: $f_{\xi}(x) = \frac{1}{\sqrt{2 \pi}\sigma} 
    e^{-\frac{(x - a)^2}{2 \sigma^2}}$
    \smallskip

    \begin{center}
        \includegraphics[width=140mm]{"21.1.png"}
    \end{center}
    \bigskip

\textbf{$(\xi_0 \sim N(0, 1))$:}
    \smallskip
    
    Функция распределения: $F_{\xi}(x) = \frac{1}{\sqrt{2 \pi}}
    \int\limits^{x}_{-\infty} e^{-\frac{t^2}{2}}dt$
    \smallskip

    Функция плотности: $f_{\xi}(x) = \frac{1}{\sqrt{2 \pi}} 
    e^{-\frac{x^2}{2}}$
    \bigskip

\textbf{Свойства нормального распределения:}
    \smallskip
    
    \begin{enumerate}
        \item{Связь $N(a, \sigma^2)$ и $N(0, 1)$:
        \bigskip
        
        Если $\xi \sim N(a, \sigma^2)$, то $\frac{\xi - a}{\sigma} = \xi_0 \sim N(0, 1)$
        \smallskip

        Если $\xi_0 \sim N(0, 1)$, то $\sigma \xi_0 + a = \xi \sim N(a, \sigma^2)$;}
        \item{Связь стандартного нормального распределения с функциями Лапласа:
        \bigskip

        С дифференциальной функцией Лапласа: 

        $\varphi(x) = \frac{1}{\sqrt{2 \pi}}e^{- \frac{x^2}{2}} \Rightarrow
        f_{\xi_0}(x) = \varphi(x)$
        \smallskip
        
        С интегральной функцией Лапласа:
        
        $0,5 + \Phi(x) = \frac{1}{\sqrt{2 \pi}} \int\limits^{0}_{-\infty} e^{-\frac{t^2}{2}}dt +
        \frac{1}{\sqrt{2 \pi}} \int\limits^{x}_{0} e^{-\frac{t^2}{2}}dt = 
        F_{\xi_0}(x)$;}
        \item{Правило трёх сигм $(3 \time \sigma)$:
        \bigskip
        
        Почти всё нормальное распределение лежит в диапазоне 
        $(a - 3 \sigma; a + 3 \sigma) \;\;\;\;\;\; \xi \sim N(a, \sigma^2)$.}
    \end{enumerate}

\subsection{Случайные векторы. Функция распределения случайного вектора,
её свойства. Дискретные и непрерывные случайные векторы.}

\textbf{Определение случайного вектора:}
    \smallskip

    Пусть $(\Omega, \mathcal{F}, P)$~--- векторное пространство. Случайным вектором
    называется вектор со случайными координатами:

    \[
        \overline{\xi} = (\xi_1, \dots, \xi_n),  
    \]

    где $\xi_i \in (\Omega, \mathcal{F}, P)_i$.
    \bigskip

\textbf{Определение функции распределения случайного вектора. Её свойства:}
    \smallskip
    
    Функцией распределения вероятностей случайного веткора называется:

    \[
        F_{\overline{\xi}}(x_1, \dots, x_n) = P\{w: \xi_1(w) < x_1, \dots,
        \xi_n(w) < x_n\} 
    \]
    \[
        F_{\xi \eta} (x, y) = P\{w: \xi(w) < x, \eta(w) < y\}  
    \]
    \bigskip

    Свойства функции распределения:
    \begin{enumerate}
        \item{$\forall x, y \in R \;\;\; 0 \leq F_{\xi \eta (x, y)} \leq 1$;}
        \item{Пусть $x_0$ фиксирован. Тогда
        
        $F_{\xi \eta} (x_0, y)$~--- неубывающая, непрерывная слева функция 
        по $y$.
        \bigskip
        
        Пусть $y_0$ фиксирован. Тогда 
        
        $F_{\xi \eta} (x, y_0)$~--- неубывающая, непрерывная слева функция 
        по $x$.}
        \item{
            \[
                \lim_{x \to +\infty} F_{\xi \eta} (x, y) = F_{\eta} (y)  
            \]
            \[
                \lim_{y \to +\infty} F_{\xi \eta} (x, y) = F_{\xi} (x)
            \]
            \[
                \lim_{x \to +\infty; y \to +\infty} F_{\xi \eta} (x, y) = 1 
            \]
            \[
                \lim_{x \to -\infty} F_{\xi \eta} (x, y) = 
                \lim_{y \to -\infty} F_{\xi \eta} (x, y) =
                \lim_{x \to -\infty; y \to -\infty} F_{\xi \eta} (x, y) = 0
            \]
        }
    \end{enumerate}

    \begin{center}
        \includegraphics[width=140mm]{"22.1.png"}
    \end{center}

\subsection{Независимые случайные величины. Критерий независимости случайных величин.}

\textbf{Определение случайных независимых
велечин:}
    \smallskip

    Случайные велечины $\xi$ и $\mu$ называются
    независимыми, если:

    \[ 
        \forall x, y \in R \;\;\;\; P\{w : \xi(w)
        < x; \mu (w) < y\} = P\{w: \xi (w) < x\} \cdot
        P\{w : \mu (w) < y\}
    \]
    \bigskip

\textbf{Критерий независимости дискретной случаной величины:}
    \smallskip

    Дискретные случайные величины $\xi$ и $\eta$ независимы, тогда
    и только тогда, когда:

    \[
        \forall i \neq j \;\;\; p_{ij} = p_i p_j  
    \]
    \bigskip

\subsection{Числовые характеристики случайной величины: Математическое ожидание
и его свойства.}    

Пусть $(\Omega, \mathcal{F}, P)$ "--- вероятностное пространство
и пусть $\xi$ "--- случайная величина на нём.

\[
\xi = \xi(w) \;\;\; P = P(w)
\]
\bigskip

\textbf{Определение математического ожидания:}
    \smallskip

    Математическим ожиданием случайной величины $\xi$ называется

    \[
        M\xi = \int\limits_{\Omega} \xi(w) d P(w)  
    \]

    Пусть для $\xi$ построена функция распределения $F_{\xi}(x) =
    P\{\xi < x\}$.

    Тогда

    \[
        M\xi = \int\limits^{\infty}_{-\infty}  x d F_{\xi} (x)  
    \]

    Для дискретной случайной величины математическое ожидание 
    находится по формуле:

    \[
        M\xi = \sum^{\infty}_{i = 1} x_i p_i  
    \]

    Для абсолютно непрерывной величины:

    \[
        M\xi = \int\limits^{\infty}_{-\infty} x f_{\xi} (x) dx  
    \]

\textbf{Свойствай математического ожидания:}
    \smallskip    

    \begin{enumerate}
        \item{$Mc = c$, $c$ "--- const;}
        \item{$Mc\xi = cM\xi$;}
        \item{$M(\xi \pm \eta) = M\xi \pm M\eta$;}
        \item{Если $\xi$ и $\eta$ независимы, то 
        
        \[
            M\xi\eta = M\xi M\eta;   
        \]
        }
        \item{Если $\xi \geq 0 \;\;\; (P\{\xi \geq 0\} = 1)$, то
        $M\xi \geq 0$;}
        \item{Неравенство Коши"=Буняковского:
        
        \[
            M|\xi\eta| \leq M|\xi|M|\eta|;  
        \]
        }
        \item{Неравенство Чебышёва:
        
        Пусть $\xi$ "--- некоторая неотрицательная величина,
        а $g(x)$, неубывающая на множестве значений $\xi$, 
        непрерывная функиция. Тогда
        
        \[
            \forall \varepsilon > 0 \;\;\; P\{\xi \geq \varepsilon\} \leq
            \frac{Mg(\xi)}{g(\varepsilon)};
        \]
        }
    \end{enumerate}

\subsection{Обобщённое неравенство Чебышёва. Следствие неравенства Чебышёва.}
    
\textbf{Неравенство Чебышёва:}
    \smallskip
            
    Пусть $\xi$ "--- некоторая неотрицательная величина,
    а $g(x)$, неубывающая на множестве значений $\xi$, 
    непрерывная функиция. Тогда
    
    \[
        \forall \varepsilon > 0 \;\;\; P\{\xi \geq \varepsilon\} \leq
        \frac{Mg(\xi)}{g(\varepsilon)};
    \]
    \bigskip

    Доказательство:
    \bigskip
    
    Рассмотрим $Mg(\xi)$:

    \[
        Mg(\xi) = \int^{\infty}_{-\infty} g(x)dF_{\xi}(x) = 
        \int^{\infty}_0 g(x) d F_{\xi}(x) \geq \int^{\infty}_{\epsilon}
        g(x)d F_{\xi}(x) \geq 
    \]
    \[
        \geq
        g(\epsilon) \int^{\infty}_{\epsilon} d F_{\xi}(x)
        = g(\epsilon) \lim_{x \to + \infty} (F_{\xi}(x) - F_{\xi}(\epsilon)) =
        g(\epsilon) (1 - P\{\xi > \epsilon\}) =
    \]
    \[
        =
        g(\epsilon)P\{\xi \geq \epsilon\}
        \leq \frac{Mg(\xi)}{g(\epsilon)}
    \]
    \bigskip

\textbf{Следствие неравенства Чебышёва:}
    \smallskip
    
    Пусть $\xi$~--- некоторая случайная величина с конечным математическим
    ожиданием. Тогда 

    \[
        \forall \epsilon > 0 \;\;\; P\{|\xi - M\xi \geq \epsilon\} \leq
        \frac{M(\xi - M\xi)^2}{\epsilon^2}
    \]

\subsection{Вычисление математического ожидания для распределений Бернулли,
биномиального распределения, распределения Пуассона, равномерного непрерывного,
показательного, нормального законов распределения.}

\textbf{Для распределения Бернулли:}
    \smallskip

    \[
        M\xi = 0 \cdot q + q \cdot p = p  
    \]
    \bigskip

\textbf{Для биномиального распределения:}
    \smallskip
    
    \[
        M\xi = M(\xi_1, \dots, \xi_n) = M\xi_1 + \dots M\xi_n = p + \dots + p = np  
    \]
    \bigskip

\textbf{Для распределения Пуассона:}
    \smallskip
    
    \[
        M\xi = \sum^{\infty}_{k = 0} k \frac{\lambda e^{-\lambda}}{k!} =
        \lambda e^{-\lambda} \sum^{\infty}_{k = 1} \frac{\lambda^{k - 1}}{(k - 1)!}=
        \lambda e^{-\lambda} e^{\lambda} = \lambda 
    \]
    \bigskip

\textbf{Для равномерного непрерывного распределения:}
    \smallskip
    
    \[
        M\xi = \int^{\infty}_{-\infty} x f(x)dx =
        \int^a_b x \frac{1}{b - a} dx = \frac{1}{b - a} \cdot \frac{b^2 - a^2}{2} =
        \frac{b + a}{2}  
    \]
    \bigskip

\textbf{Для показательного распределения:}
    \smallskip
    
    \[
        M\xi = \int^{\infty}_0 x \cdot \lambda e^{-\lambda x} dx =
        -x e^{-\lambda x} |^{\infty}_0 + \int^{\infty}_0 \lambda e^{-\lambda x} dx =
    \]
    \[
        =
        - (\lim_{x \to +\infty} x e^{-\lambda x} - e^0) + \frac{-1}{\lambda}
        (\lim_{x \to +\infty} x e^{-\lambda x} - e^0) = \frac{1}{\lambda}  
    \]
    \bigskip

\textbf{Для нормального распределения:}
    \smallskip
    
    \[
        M\xi_0 = \int^{\infty}_{-\infty} x \cdot e^{-\frac{x^2}{2}} dx =
        \int^{0}_{-\infty} x \cdot e^{-\frac{x^2}{2}} dx +
        \int^{\infty}_{0} x \cdot e^{-\frac{x^2}{2}} dx = 0
    \]
    \[
        M\xi = M(\sigma \xi_0 + a) = a  
    \]

\subsection{Числовые характеристики случайных величин: Начальные,
центральные и смешанные моменты. Дисперсия и её свойства. Ковариация и её свойства.
Коэффициент корреляции и его свойства.}

\begin{center}
    \includegraphics[width=140mm]{"27.1.png"}
\end{center}

\textbf{Определение дисперсии случайной величины:}
    \smallskip

    Пусть $\xi$ "--- случайная величина и $|M\xi| < +\infty$.
    \bigskip

    Дисперсией случайной величины $\xi$ называется число 

    \[
    D\xi = M(\xi - M\xi)^2
    \]

\textbf{Свойства дисперсии:}
    \smallskip

    \begin{enumerate}
        \item{$D\xi \geq 0$;}
        \item{$Dc = 0$;}
        \item{$Dc\xi = c^2 D\xi$;}
        \item{Если случайные величины $\xi$ и $\eta$ независимы, то
        
        \[
            D(\xi \pm \eta) = D\xi \pm D\eta;
        \]
        }
        \item{$D\xi = M\xi^{2} - (M\xi)^2$;}
        \item{Для произвольных случайных величин $\xi$ и $\eta$ с
        $M|\xi| < + \infty$ и $M|\eta| < + \infty$ верно
        
        \[
            D(\xi \pm \eta) = D\xi \pm D\eta \pm {\text{cov}(\xi, \eta)},
        \]

        где ${\text{cov}(\xi, \eta)} = M(\xi - M\xi)(\eta - M\eta)$ "--- ковариация.
        }
    \end{enumerate}

    \begin{center}
        \includegraphics[width=140mm]{"27.2.png"}
    \end{center}

\end{document}