\documentclass[14pt]{extarticle}

\usepackage[T2A]{fontenc}
\usepackage[cp1251]{inputenc}
\usepackage[russian]{babel}

\usepackage{graphicx}

\usepackage{amsmath, amsthm, amsfonts, amssymb}
\usepackage[unicode, hidelinks]{hyperref}

\usepackage[
    a4paper, mag=1000,
    left=2.5cm, right=1.5cm, top=2cm, bottom=2cm, bindingoffset=0cm,
    headheight=0cm, footskip=1cm, headsep=0cm
]{geometry}

\usepackage{tempora}

\usepackage{chngcntr} % equation counter manipulations

% reset equation counter at each section
\counterwithin*{equation}{section}

\graphicspath{{images/}}

\begin{document}
    
\tableofcontents

\newpage

\section{Глава 1}

\subsection{Случайные события, классификация событий, операции над ними.}

\textbf{Определение случайного события:}
    \smallskip

    Пусть $\Omega$ "--- множество элементарных
    исходов эксперимента. Случайным событием
    называется любое подмножество множества
    $\Omega$.
    \bigskip

\textbf{Определение достоверного события:}
    \smallskip

    Достоверным событием называется событие
    $\Omega$, которому благоприятствует
    каждый исход эксперимента.
    \bigskip

\textbf{Определение невозможного события:}
    \smallskip
    
    Невозможным событием называется пустое
    множество, которому не благоприятствует ни
    один исход эксперимента.
    \bigskip

\textbf{Определение суммы событий:}
    \smallskip

    Суммой событий $A$ и $B$ называется событие $C = A \cup B$,
    которому благоприятствуют исходы, принадлежащие хоть одному
    из событий $A$ или $B$.
    \bigskip

\textbf{Определение произведения событий:}
    \smallskip
    
    Произведением событий $A$ и $B$ называется событие 
    $C = A \cap B$, которому благоприятствуют исходы 
    и события $A$, и события $B$.
    \bigskip

\textbf{Определение несовместных событий:}
    \smallskip

    Случайные события $A$ и $B$ называются несовместными, если 
    $A \cap B = \varnothing$.
    \bigskip

\textbf{Определение противоположного события:}
    \smallskip
    
    Событием, противоположным событию 
    $A$ называется событие $\overline{A}$, которое
    состоит из исходов, не благоприятствующих $A$.
    \bigskip

\subsection{Определения: кольцо, алгебра, $\sigma$"=алгебра, минимальная 
$\sigma$"=алгебра над классом $K$. Борелевская $\sigma$"=алгебра.}     

\textbf{Определение кольца:}
    \smallskip

    Кольцом $\mathbf{R}$ называется непустой класс множества
    замкнутый относительно операций сложения и взятия разности.
    \bigskip

\textbf{Определение алгебры:}
    \smallskip

    Алгеброй $\mathcal{A}$ называется непустой класс множества
    замкнутый относительно сложения и отрицания.
    \bigskip
    
\textbf{Определение $\sigma$"=алгебры:}
    \smallskip

    $\sigma$"=алгебра $\mathcal{F}$~--- это непустой класс множества
    замкнутый относительно счётного количества сумм и отрицаний:

    \begin{enumerate}
        \item{Если $A \in \mathcal{F}$, то $\overline{A} \in \mathcal{F}$;}
        \item{$\Omega \in \mathcal{F}$;}
        \item{Если $\{A_i\}^{\infty}_{i = 1} \in \mathcal{F}$,
        то $\bigcup^{\infty}_{i = 1} A_i \in \mathcal{F}$.}
    \end{enumerate}
    \bigskip

\textbf{Определение $\sigma$"=алгебры событий:}
    \smallskip

    Сигма алгеброй событий называется множество
    $\mathcal{F}$ подмножеств $A \subset \Omega$,
    удовлетворяющее условиям:

    \begin{enumerate}
        \item{если $A \in \mathcal{F}$, то
        $\overline{A} \in \mathcal{F}$;}
        \item{$\Omega \in \mathcal{F}$;}
        \item{если $\{A\}_{i = 1}^{\infty} \in
        \mathcal{F}$, то
        $\bigcup\limits_{i = 1}^{\infty} A_i \in 
        \mathcal{F}$.}
    \end{enumerate}
    \bigskip    

\textbf{Определение минмальной $\sigma$"=алгебры над классом $K$:}
    \smallskip
    
    Пусть $K$~--- некоторый класс подмножеств из $\Omega$. $\sigma$"=алгебра
    $\sigma(K)$ называется наименьшей $\sigma$"=алгеброй, содержащей
    класс $K$, если $K \in \sigma(K)$; любая $\sigma$"=алгебра $\mathcal{F}$,
    которая содержит $K \;\; (K \subset \mathcal{F})$, содержит и 
    $\sigma(K) \subset \mathcal{F}$.
    \bigskip

\textbf{Определение Борелевской $\sigma$"=алгебры:}
    \smallskip
    
    Борелевской $\sigma$"=алгеброй $\beta$ называется минимальная $\sigma$"=алгебра
    над классом полуинтервалов $K = \{[a, b]\}$ из $R$, то есть:

    \[
        \Omega = (-\infty, \infty) = R \;\;\; K = \{[a, b), [a, +\infty),
        (a, +\infty), (-\infty, b), (-\infty, b], (a, b]\}.  
    \]

\end{document}