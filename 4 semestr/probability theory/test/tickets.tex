\documentclass[14pt]{extarticle}

\usepackage[T2A]{fontenc}
\usepackage[cp1251]{inputenc}
\usepackage[russian]{babel}

\usepackage{graphicx}

\usepackage{amsmath, amsthm, amsfonts, amssymb}
\usepackage[unicode, hidelinks]{hyperref}

\usepackage[
    a4paper, mag=1000,
    left=2.5cm, right=1.5cm, top=2cm, bottom=2cm, bindingoffset=0cm,
    headheight=0cm, footskip=1cm, headsep=0cm
]{geometry}

\usepackage{tempora}

\usepackage{chngcntr} % equation counter manipulations

% reset equation counter at each section
\counterwithin*{equation}{section}
\counterwithin*{equation}{subsection}

\graphicspath{{images/}}

\usepackage{cases}

\begin{document}
    
\tableofcontents

\newpage

\section{Глава 1}

\subsection{Случайные события, классификация событий, операции над ними.}

\textbf{Определение случайного события:}
    \smallskip

    Пусть $\Omega$ "--- множество элементарных
    исходов эксперимента. Случайным событием
    называется любое подмножество множества
    $\Omega$.
    \bigskip

\textbf{Определение достоверного события:}
    \smallskip

    Достоверным событием называется событие
    $\Omega$, которому благоприятствует
    каждый исход эксперимента.
    \bigskip

\textbf{Определение невозможного события:}
    \smallskip
    
    Невозможным событием называется пустое
    множество, которому не благоприятствует ни
    один исход эксперимента.
    \bigskip

\textbf{Определение суммы событий:}
    \smallskip

    Суммой событий $A$ и $B$ называется событие $C = A \cup B$,
    которому благоприятствуют исходы, принадлежащие хоть одному
    из событий $A$ или $B$.
    \bigskip

\textbf{Определение произведения событий:}
    \smallskip
    
    Произведением событий $A$ и $B$ называется событие 
    $C = A \cap B$, которому благоприятствуют исходы 
    и события $A$, и события $B$.
    \bigskip

\textbf{Определение несовместных событий:}
    \smallskip

    Случайные события $A$ и $B$ называются несовместными, если 
    $A \cap B = \varnothing$.
    \bigskip

\textbf{Определение противоположного события:}
    \smallskip
    
    Событием, противоположным событию 
    $A$ называется событие $\overline{A}$, которое
    состоит из исходов, не благоприятствующих $A$.
    \bigskip

\subsection{Определения: кольцо, алгебра, $\sigma$"=алгебра, минимальная 
$\sigma$"=алгебра над классом $K$. Борелевская $\sigma$"=алгебра.}     

\textbf{Определение кольца:}
    \smallskip

    Кольцом $\mathbf{R}$ называется непустой класс множества
    замкнутый относительно операций сложения и взятия разности.
    \bigskip

\textbf{Определение алгебры:}
    \smallskip

    Алгеброй $\mathcal{A}$ называется непустой класс множества
    замкнутый относительно сложения и отрицания.
    \bigskip
    
\textbf{Определение $\sigma$"=алгебры:}
    \smallskip

    $\sigma$"=алгебра $\mathcal{F}$~--- это непустой класс множества
    замкнутый относительно счётного количества сумм и отрицаний:

    \begin{enumerate}
        \item{Если $A \in \mathcal{F}$, то $\overline{A} \in \mathcal{F}$;}
        \item{$\Omega \in \mathcal{F}$;}
        \item{Если $\{A_i\}^{\infty}_{i = 1} \in \mathcal{F}$,
        то $\bigcup^{\infty}_{i = 1} A_i \in \mathcal{F}$.}
    \end{enumerate}
    \bigskip

\textbf{Определение $\sigma$"=алгебры событий:}
    \smallskip

    Сигма алгеброй событий называется множество
    $\mathcal{F}$ подмножеств $A \subset \Omega$,
    удовлетворяющее условиям:

    \begin{enumerate}
        \item{если $A \in \mathcal{F}$, то
        $\overline{A} \in \mathcal{F}$;}
        \item{$\Omega \in \mathcal{F}$;}
        \item{если $\{A\}_{i = 1}^{\infty} \in
        \mathcal{F}$, то
        $\bigcup\limits_{i = 1}^{\infty} A_i \in 
        \mathcal{F}$.}
    \end{enumerate}
    \bigskip    

\textbf{Определение минмальной $\sigma$"=алгебры над классом $K$:}
    \smallskip
    
    Пусть $K$~--- некоторый класс подмножеств из $\Omega$. $\sigma$"=алгебра
    $\sigma(K)$ называется наименьшей $\sigma$"=алгеброй, содержащей
    класс $K$, если $K \in \sigma(K)$; любая $\sigma$"=алгебра $\mathcal{F}$,
    которая содержит $K \;\; (K \subset \mathcal{F})$, содержит и 
    $\sigma(K) \subset \mathcal{F}$.
    \bigskip

\textbf{Определение Борелевской $\sigma$"=алгебры:}
    \smallskip
    
    Борелевской $\sigma$"=алгеброй $\beta$ называется минимальная $\sigma$"=алгебра
    над классом полуинтервалов $K = \{[a, b]\}$ из $R$, то есть:

    \[
        \Omega = (-\infty, \infty) = R \;\;\; K = \{[a, b), [a, +\infty),
        (a, +\infty), (-\infty, b), (-\infty, b], (a, b]\}.  
    \]
    \bigskip

\subsection{Теорема Каратеодори.}

    Пусть $Q(A)$~--- счётно аддитивная вероятностная мера на алгебре $\mathcal{A}$.
    Тогда существует единственная счётно аддитивная вероятностная мера $P(A)$, заданная
    на минимальной $\sigma$"=алгебре $\mathcal{F}$ и являющаяся её продолжением,
    то есть $\forall A \in \mathcal{A} \;\;\; P(A) = Q(A).$
    \bigskip

\subsection{Определения: мера, конечно"=аддитивная, счётно"=аддитивная мера}

Пусть $\Omega$~--- множество элементарных исходов эксперимента. Некоторое 
его подмножество $A \subset \Omega$ называется случайным событием.
\bigskip

\textbf{Определение конечно аддитивной
вероятностной меры}
    \smallskip

    Конечно аддитивной вероятностной мерой $Q(A)$
    называется функция множества $Q: \mathcal{A}
    \rightarrow [0; 1]$, такая, что:

    \begin{enumerate}
        \item{$\forall A \in \mathcal{A} \;\;\;
        Q(A) \geq 0$;}
        \item{$Q(\Omega) = 1$;}
        \item{$\forall A, B \in \mathcal{A} : 
        \;\;\; A \cap B = \varnothing \;\;\;
        Q(A \cup B) = Q(A) + Q(B) \;\;\;
        Q\left(\bigcup\limits_{i = 1}^{\infty} A_i\right) = \sum_{i = 1}^{\infty}
        Q(A_i)$.}
    \end{enumerate}
    \bigskip  

\textbf{Определение счётно аддитивной 
вероятностной меры:}
    \smallskip

    Счётно аддитивно вероятностной мерой $P(A)$
    называется функция множества $P: \mathcal{F}
    \rightarrow [0; 1]$, такая, что:

    \begin{enumerate}
        \item{$\forall A \in \mathcal{F} \;\;\;
        P(A) \geq 0$;}
        \item{$P(\Omega) = 1$;}
        \item{$\forall \{A_i\}_{i = 1}^{\infty}
        \in \mathcal{F} : \;\;\; \forall
        i \neq j \;\;\; A_i \bigcap A_j = 
        \varnothing \;\;\; P\left(\bigcup\limits_{i = 1}^{
        \infty} A_i\right) = \sum_{i = 1}^{\infty}
        P(A_i)$.}
    \end{enumerate}
    \bigskip  

\subsection{Построение меры Лебега. Верхняя мера Лебега, нижняя мера Лебега, мера
Лебега. Измеримое по Лебегу множество.}    

Пусть $P = <a, b> \times <c, d>$. $P \subset R^2$~--- прямоугольник.
\bigskip

Мерой прямоугольника назовём $m(P)$, где

$m(P) = (b - a)(d - c)$

\begin{center}
    \includegraphics[width=100mm]{"5.1.png"}
\end{center}
\bigskip

Множество $A$ назовём эелментарным, если оно представимо в виде
суммы прямоугольников хотя бы 1 способом:

\[
    A = \bigcup P_k,  
\]

где $\{P_k\}$~--- покрытие.
\bigskip

Мерой элементарного сножества $A$ называется 

\[
    m^{'}(A) = \sum m(P_k),  
\]

где $\{P_k\}$~--- разбиение $A$, то есть $\forall j \neq k \;\;\; P_k \cap P_j =
\varnothing$.
\bigskip

Рассмотрим множество $E = [0; 1] \times [0; 1]$

\begin{center}
    \includegraphics[width=100mm]{"5.2.png"}
\end{center}
\bigskip

\textbf{Определение верхней меры Лебега:}
    \smallskip    

    Пусть $A$~--- некоторое множество. Рассмотрим $\{P_k\}$, такое, что:

    \[
        A \subset P_k
    \]

    \begin{center}
        \includegraphics[width=100mm]{"5.3.png"}
    \end{center}

    Верхней мерой Лебега называется 

    \[
        \mu^{*}(A) = \inf_{\{P_k\}} \sum m(P_k)
    \]
    \bigskip

\textbf{Определение нижней меры Лебега:}
    \smallskip
    
    Рассмотрим множество $E \backslash A. (m(E) = 1)$

    \begin{center}
        \includegraphics[width=100mm]{"5.4.png"}
    \end{center}

    Нижней мерой Лебега называется:

    \[
        \mu_{*}(A) = 1 - \mu^{*}(E \backslash A)  
    \]
    \bigskip
    
\textbf{Определение меры Лебега и измеримого по Лебегу множества:}
    \smallskip
    
    Говорят, что множество $A$ измеримо по Лебегу, если $\mu^{*}(A) =
    \mu_{*}(A) = \mu (A)$.

    Величина $\mu(A)$~--- называется мерой Лебега множества $A$.

\subsection{Вероятностная мера, её свойства, непрерывность вероятностной меры.}

\textbf{Определение вероятностной меры:}
    \smallskip

    Вероятностной мерой называется функция $P : \mathcal{F} \to [0, 1]$,
    удовлетворяющая условиям:

    \begin{enumerate}
        \item{$P(\Omega) = 1$;}
        \item{$\forall A \in \mathcal{F} \;\;\; P(A) \geq 0$;}
        \item{$\forall \{A_i\}^{\infty}_{i = 1} \in \mathcal{F}$,
        такой, что $\forall i \neq j \;\;\; A_i \cap A_j = \varnothing$,
        $P\left(\bigcup\limits^{\infty}_{i = 1}A_i\right) = \sum\limits^{\infty}_{i = 1} P(A_i)$.}
    \end{enumerate}
    \bigskip

\textbf{Свойства вероятностной меры:}    
    \smallskip

    \begin{enumerate}
        \item{
        \begin{center}
            \includegraphics[width=100mm]{"6.1.png"}
        \end{center}
        }
        \item{
        \begin{center}
            \includegraphics[width=100mm]{"6.2.png"}
        \end{center}
        }
        \item{
        \begin{center}
            \includegraphics[width=100mm]{"6.3.png"}
        \end{center}
        }
        \item{
        \begin{center}
            \includegraphics[width=100mm]{"6.4.png"}

            \includegraphics[width=120mm]{"6.5.png"}
        \end{center}
        }
    \end{enumerate}

\subsection{Классическое вероятностное пространство. Классическое определение
вероятности.}

Вероятностной моделью стохастического эксперимента называется тройка
$(\Omega, \mathcal{F}, P)$, где $\Omega$~--- множество элементарных
исходов экмперимента, $\mathcal{F}$~--- алгебра событий, 
$P$~--- вероятностная мера.

$(\Omega, \mathcal{F}, P)$~--- вероятностное пространство.
\bigskip

\textbf{Определение классического вероятностного пространства:}
    \smallskip

    Классическим вероятностым пространством, называется вероятностное
    пространство $(\Omega, \mathcal{F}, P)$, в конечном множестве 
    элементарных исходов которого все элементарные исходы равновозможны.
    \bigskip

    Построим вероятностную меру:
    \bigskip

    Пусть $\Omega = \{w1, \dots, w_n\}$.

    Рассмотрим $\{ A_i \}^{n}_{i = 1}$, где $A_i = \{w_i\}$. Тогда

    \[
        A_i \cap A_j = \varnothing \;\;\; \bigcup^{n}_{i = 1} A_i = \Omega.  
    \]

    \[
        1 = P(\Omega) = P(\bigcup^{n}_{i = 1} A_i) = \sum^{n}_{i = 1} P(A_i) =
        | P(A_i) = P(w_i) = p | = \sum^{n}_{i = 1} p \Rightarrow
    \]
    \[
        \Rightarrow p = \frac{1}{n} \Rightarrow \forall w_i \;\;\; P(w_i) =
        \frac{1}{n}  
    \]

    Пусть $A = \{w_{i_1}, \dots, w_{i_k}\}$ $0 \leq k \leq n$. Тогда
    вероятностная мера в классическом вероятностном пространстве имеет
    вид 

    \[
        P(A) = P \left( \bigsqcup^{k}_{j = 1} w_{ij} \right) =
        \sum^{k}_{j = 1} P(w_{ij}) = \sum^{k}_{j = 1} \frac{1}{n} = \frac{k}{n},  
    \]
    \bigskip

    \textbf{$P(A) = \frac{k}{n}$~--- называется классической вероятностью},

    где $k$~--- количество благоприятных $A$ элементарных исходов,
    $n$~--- количество элементарных исходов эксперимента.

\subsection{Дискретное вероятностное пространство.}

\textbf{Определение дискретного вероятностного пространства:}
    \smallskip

    Дискретным вероятностным пространством называется вероятностное
    пространство $(\Omega, \mathcal{F}, P)$, такое, что 
    $\Omega$~--- конечное или счётное множество неравновозможных исходов.
    \bigskip

    Вероятностную меру зададим числами $p_i = P(w_i) > 0$, такими, что
    $\sum^{\infty}_{i = 1} p_i = 1$. Тогда $\forall A \in \mathcal{F}$
    веротность вычисляется как $P(A) = P(\bigcup\limits_{w_i \in A} w_i) =
    \sum\limits_{w_i \in A} P (w_i)$.

\subsection{Условная вероятность. Теорема умножения вероятностей.}    

\textbf{Определение условной вероятности:}
    \smallskip

    Пусть $(\Omega, \mathcal{F}, P)$~--- вероятностное пространство и
    $A, B \in \mathcal{F}; \;\;\; P(B) > 0$. 

    Условной вероятностью события $A$ при условии, что наступило
    событие $B$ называется число:

    \[
        P(A|B) = \frac{P(A \cap B)}{P(B)}.  
    \]
    \bigskip

\textbf{Теорема умножения вероятностей:}
    \smallskip
    
    Пусть $A$ и $B$~--- случайные события и $P(B) > 0$. Тогда 

    \[
        P(A \cap B) = P(B) \cdot P(A|B)  
    \]
    \bigskip

    Пусть $A_1, A_2, A_3$~--- случайные события и $P(A_1) > 0$ и $P(A_1 \cap A_2) > 0$.
    Тогда

    \[
        P(A_1 \cap A_2 \cap A_3) = P(A_1) \cdot P(A_2|A_1) \cdot P(A_3|A_1 \cap A_2)  
    \]

\subsection{Формулы полной вероятности и Байеса.}

\textbf{Теорема (формула полной вероятности):}
    \smallskip

    Пусть $(\Omega, \mathcal{F}, P)$~--- вероятностное пространство и
    $\{A_i\}^{\infty}_{i = 1} \in \mathcal{F}$~--- полная группа
    попарно несовместных событий; $P(A_i) \geq 0$. Пусть $A \in \mathcal{F}$~---
    неполное событие $P(A|A_i) \geq 0$. Тогда $P(A) = \sum^{\infty}_{i = 1} P(A_i)
    \cdot P(A|A_i)$.
    \bigskip

    Доказательство:
    \bigskip

    $$P(A) = P(A \cap \Omega) = P(A \cap (\sqcup^{\infty}_{i = 1} A_i)) =
    P(\sqcup^{\infty}_{i = 1}(A \cap A_i)) \Rightarrow$$

    $$\Rightarrow \sum^{\infty}_{i = 1} P (A \cap A_i) = \sum^{\infty}_{i = 1}
    P(A_i)P(A|A_i)$$.
    \bigskip

\textbf{Теорема (формула Байеса):}
    \smallskip
    
    Пусть $\{A_i\}^{\infty}_{i = 1} \in \mathcal{F}$~--- полная группа
    попарно несовместных событий и пусть для некоторого $P(A) > 0$. Тогда

    \[
        \forall i = \overline{1, \infty} \;\;\; P(A_i|A) = 
        \frac{P(A_i)P(A|A_i)}{P(A)}  
    \]
    \bigskip

    Доказательство:
    \bigskip

    \[
        P(A_i|A) = \frac{P(A \cap A_i)}{P(A)} = \frac{P(A_i) \cdot P(A|A_i)}{P(A)}  
    \]

\subsection{Независимость событий. Независимость в совокупности.}

\textbf{Определение независимости событий:}
    \smallskip

    Случайные события $A$ и $B$ называются независимыми, если:

    \[
        P(A \cap B) = P(A) \cdot P(B)   
    \]
    \bigskip

\textbf{Определение независимости в совокупности:}
    \smallskip

    $\{A_i\}^{n}_{i = 1}$~--- называются независимыми в совокупности, если

    \[
        \forall 2 \leq k \leq n \;\;\; P (\bigcap^{k}_{j = 1}A_{ij}) =
        \prod^{k}_{j = 1} P(A_{ij}) 
    \]

\subsection{Теорема о независимости противоположных событий. Критерий
независимости случайных событий.}

\textbf{Теорема о независимости противоположных событий:}
    \smallskip

    Пусть $A$ и $B$~--- независимы. Тогда события $A$ и $\overline{B}$,
    $\overline{A}$ и $B$. $\overline{A}$ и $\overline{B}$~--- попарно
    независимы.
    \bigskip

    Доказательство:
    \bigskip

    Рассмотрим $A$ и $\overline{B}$. Тогда $P(A \cap 
    \overline{B})$. Можем заметить, что $P(A \cap \overline{B}) = 
    P(A \backslash (A \cap B)) = P(A) - P(A \cap B) = P(A) - P(A) P(B) =
    P(A)(1 - P(B)) = P(A) P(\overline{B})$.

    Остальные случаи аналогичны.
    \bigskip

\textbf{Критерий независимости случайных событий:}
    \smallskip

    Пусть $A$ и $B$ такие, что $P(B) > 0$. Тогда Случайные
    события $A$ и $B$ независимы, тогда и только тогда, когда:

    \[
        P(A|B) = P(A)  
    \]
    \bigskip

    Доказательство:
    \bigskip

    Необходимость:
    \smallskip

    Пусть $A$ и $B$ независимы:

    \[
        P(A|B) = \frac{P(A \cap B)}{P(B)} = \frac{P(A) \cdot P(B)}{P(B)} = P(A)  
    \]
    \bigskip

    Достаточность:
    \smallskip

    Пусть выполняется: $P(A|B) = P(A)$. Тогда из определения условной вероятности
    следует, что $P(A \cap B) = P(A|B) \cdot P(A) = P(A) \cdot P(B)$,
    то есть выполняется определение.

\end{document}